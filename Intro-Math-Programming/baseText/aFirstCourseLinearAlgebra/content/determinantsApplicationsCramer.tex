\subsection{Cramer's Rule}

%Another context in which the formula given in Theorem
%\ref{thm:inverseanddeterminant} is important is \textbf{Cramer's
%Rule}.  
Recall that we can represent a system of linear equations in
the form $AX=B$, where the solutions to this system are given by $X$.
Cramer's Rule gives a formula for the solutions $X$ in the special
case that $A$ is a square invertible matrix. Note this rule does not
apply if you have a system of equations in which there is a different
number of equations than variables (in other words, when $A$ is not
square), or when $A$ is not invertible.

Suppose we have a system of equations given by $AX=B$, and we want to find solutions $X$ which satisfy 
this system.
Then recall that if $A^{-1}$ exists,
\begin{eqnarray*}
AX&=&B \\
A^{-1}\left(AX\right)&=&A^{-1}B \\
\left(A^{-1}A\right)X&=&A^{-1}B \\
IX&=&A^{-1}B\\
X &=& A^{-1}B
\end{eqnarray*}
Hence, the solutions $X$ to the system are given by $X=A^{-1}B$. 
Since we assume that $A^{-1}$ exists, we can use the
formula for $A^{-1}$ given above. Substituting this formula into the equation for $X$, we have 
\begin{equation*}
X=A^{-1}B=\frac{1}{\det \left( A\right) }\func{adj}\left( A\right)B
\end{equation*}
Let $x_i$ be the $i^{th}$ entry of $X$ and $b_j$ be the $j^{th}$ entry of $B$.
Then this equation becomes
\begin{equation*}
x_i = \sum_{j=1}^{n}\leftB a_{ij}\rightB^{-1}b_{j}=\sum_{j=1}^{n}\vspace{0.05in}\frac{1}
{\det \left( A\right) } \func{adj}\left( A\right) _{ij}b_{j}
\end{equation*}
where $\func{adj}\left(A\right)_{ij}$ is the $ij^{th}$ entry of $\func{adj}\left(A\right)$.

By the formula for the expansion of a determinant along a column,
\begin{equation*}
x_{i}=\vspace{0.05in}\frac{1}{\det \left( A\right) }\det \leftB
\begin{array}{ccccc}
\ast & \cdots & b_{1} & \cdots & \ast \\
\vdots &  & \vdots &  & \vdots \\
\ast & \cdots & b_{n} & \cdots & \ast
\end{array}
\rightB 
\end{equation*}
where here the $i^{th}$ column of $A$ is replaced with the column vector 
$\leftB b_{1}\cdots \cdot ,b_{n}\rightB ^{T}$. The determinant of this
modified matrix is taken and divided by $\det \left( A\right) $. This
formula is known as Cramer's rule.
\index{Cramer's rule}

We formally define this method now. 

\begin{procedure}{Using Cramer's Rule}{cramersrule}
Suppose $A$ is an $n\times n$ invertible matrix and we wish to solve the system 
$AX=B$ for $X
=\leftB x_{1},\cdots ,x_{n}\rightB ^{T}.$ Then Cramer's rule says
\begin{equation*}
x_{i}=
\vspace{0.05in}\frac{\det \left(A_{i}\right)}{\det \left(A\right)}
\end{equation*}
where $A_{i}$ is the matrix obtained by replacing the $i^{th}$ column of $A$
with the column matrix
\begin{equation*}
B = 
\leftB
\begin{array}{c}
b_1 \\
\vdots \\
b_n
\end{array}
\rightB
\end{equation*} 
\end{procedure}

We illustrate this procedure in the following example.

\begin{example}{Using Cramer's Rule}{cramersrule}
Find $x,y,z$ if
\begin{equation*}
\leftB
\begin{array}{rrr}
1 & 2 & 1 \\
3 & 2 & 1 \\
2 & -3 & 2
\end{array}
\rightB \leftB
\begin{array}{c}
x \\
y \\
z
\end{array}
\rightB =\leftB
\begin{array}{r}
1 \\
2 \\
3
\end{array}
\rightB 
\end{equation*}
\end{example}

\begin{solution} We will use method outlined in Procedure \ref{proc:cramersrule} to find the values for
$x,y,z$ which give the solution to this system. 
Let
\begin{equation*}
B = 
\leftB 
\begin{array}{r}
1 \\
2 \\
3
\end{array}
\rightB 
\end{equation*}

In order to find $x$, we calculate
\begin{equation*}
x =
\vspace{0.05in}\frac{\det \left(A_{1}\right)}{\det \left(A\right)}
\end{equation*}
where $A_1$ is the matrix obtained from replacing the first column of $A$ with $B$.

Hence, $A_1$ is given by 
\begin{equation*}
A_1 = 
\leftB
\begin{array}{rrr}
1 & 2 & 1 \\
2 & 2 & 1 \\
3 & -3 & 2
\end{array}
\rightB
\end{equation*}

Therefore,
\begin{equation*}
x=
\vspace{0.05in}\frac{\det \left(A_{1}\right)}{\det \left(A\right)}
=
\vspace{0.05in}\frac{\left|
\begin{array}{rrr}
1 &  2 & 1 \\
2 &  2 & 1 \\
3 & -3 & 2
\end{array}
\right| }{\left|
\begin{array}{rrr}
1 & 2 & 1 \\
3 & 2 & 1 \\
2 & -3 & 2
\end{array}
\right| }=\vspace{0.05in}\frac{1}{2}
\end{equation*}

Similarly, to find $y$ we construct $A_2$ by replacing the second column of $A$ with $B$. Hence, $A_2$ is given by
\begin{equation*}
A_2
=
\leftB
\begin{array}{rrr}
1 & 1 & 1 \\
3 & 2 & 1 \\
2 & 3 & 2
\end{array}
\rightB
\end{equation*}

Therefore, 
\begin{equation*}
y=\vspace{0.05in}\frac{\det \left(A_{2}\right)}{\det \left(A\right)} = \vspace{0.05in}\frac{\left|
\begin{array}{rrr}
1 & 1 & 1 \\
3 & 2 & 1 \\
2 & 3 & 2
\end{array}
\right| }{\left|
\begin{array}{rrr}
1 & 2 & 1 \\
3 & 2 & 1 \\
2 & -3 & 2
\end{array}
\right| }=-\vspace{0.05in}\frac{1}{7}
\end{equation*}

Similarly, $A_3$ is constructed by replacing the third column of $A$ with $B$. Then, $A_3$ is given by
\begin{equation*}
A_3
=
\leftB
\begin{array}{rrr}
1 & 2 & 1 \\
3 & 2 & 2 \\
2 & -3 & 3
\end{array}
\rightB
\end{equation*}

Therefore, $z$ is calculated as follows. 

\begin{equation*}
z=
\vspace{0.05in}\frac{\det \left(A_{3}\right)}{\det \left(A\right)}
=
\frac{\left|
\begin{array}{rrr}
1 & 2 & 1 \\
3 & 2 & 2 \\
2 & -3 & 3
\end{array}
\right| }{\left|
\begin{array}{rrr}
1 & 2 & 1 \\
3 & 2 & 1 \\
2 & -3 & 2
\end{array}
\right| }=\frac{11}{14}
\end{equation*}
\end{solution}

Cramer's Rule gives you another tool to consider when solving a system of linear equations.

%We can also use Cramer's Rule for systems of non linear equations. Consider the following system 
%where the matrix $A$ has functions rather than numbers for entries. 
%
%\begin{example}{Use Cramer's Rule for Non-Constant Matrix}{cramersrulenonconstantmatrix}
%Solve for $z$ if
%\begin{equation*}
%\leftB
%\begin{array}{ccc}
%1 & 0 & 0 \\
%0 & e^{t}\cos t & e^{t}\sin t \\
%0 & -e^{t}\sin t & e^{t}\cos t
%\end{array}
%\rightB \leftB
%\begin{array}{c}
%x \\
%y \\
%z
%\end{array}
%\rightB =\leftB
%\begin{array}{c}
%1 \\
%t \\
%\vspace{0.05in}t^{2}
%\end{array}
%\rightB
%\end{equation*}
%\end{example}
%
%\begin{solution} We are asked to find the value of $z$ in the solution. We will solve using Cramer's rule.
% Thus
%\begin{equation*}
%z=\vspace{.05in} \frac{\left|
%\begin{array}{ccc}
%1 & 0 & 1 \\
%0 & e^{t}\cos t & t \\
%0 & -e^{t}\sin t & t^{2}
%\end{array}
%\right| }{\left|
%\begin{array}{ccc}
%1 & 0 & 0 \\
%0 & e^{t}\cos t & e^{t}\sin t \\
%0 & -e^{t}\sin t & e^{t}\cos t
%\end{array}
%\right| }=\allowbreak t\left( \left( \cos t\right) t+\sin t\right) e^{-t}
%\end{equation*}
%\end{solution}