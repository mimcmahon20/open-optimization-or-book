\section{$LU$ Factorization}

An $LU$ factorization of a matrix involves writing the given matrix as the
product of a lower triangular matrix $L$ which has the main diagonal consisting
entirely of ones, and an upper triangular matrix $U$ in the indicated
order. This is the version discussed here but it is sometimes the case that
the $L$ has numbers other than 1 down the main diagonal. It is still a
useful concept. The $L$ goes with ``lower'' and the $U$ with ``upper''.


It turns out many matrices can be written in this way and when this is
possible, people get excited about slick ways of solving the system of
equations, $AX=B$. It is for this reason that you want to
study the $LU$ factorization. It allows you to work only with
triangular matrices. It turns out that it takes about half as many
operations to obtain an $LU$ factorization as it does to find the row
reduced echelon form.

First it should be noted not all matrices have an $LU$ factorization and so
we will emphasize the techniques for achieving it rather than formal proofs.
\index{LU decomposition!non existence}
\index{LU factorization}

\begin{example}{A Matrix with NO $LU$ factorization}{}
Can you write $\leftB
\begin{array}{rr}
0 & 1 \\
1 & 0
\end{array}
\rightB $ in the form $LU$ as just described?
\end{example}

\begin{solution}
To do so you would need 
\begin{equation*}
\leftB 
\begin{array}{rr}
1 & 0 \\ 
x & 1
\end{array}
\rightB \leftB 
\begin{array}{rr}
a & b \\ 
0 & c
\end{array}
\rightB =\allowbreak \leftB 
\begin{array}{cc}
a & b \\ 
xa & xb+c
\end{array}
\rightB =\leftB 
\begin{array}{rr}
0 & 1 \\ 
1 & 0
\end{array}
\rightB .
\end{equation*}

Therefore, $b=1$ and $a=0.$ Also, from the bottom rows, $xa=1$ which can't
happen and have $a=0.$ Therefore, you can't write this matrix in the form $%
LU.$ It has no $LU$ factorization. This is what we mean above by saying the
method lacks generality.

Nevertheless the method is often extremely useful, and we will describe
below one the many methods used to produce an $LU$ factorization when
possible.
\end{solution}