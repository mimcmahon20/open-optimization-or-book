%By Kevin Cheung
%The book is licensed under the
%\href{http://creativecommons.org/licenses/by-sa/4.0/}{Creative Commons
%Attribution-ShareAlike 4.0 International License}.
%
%This file has been modified by Robert Hildebrand 2020.  
%CC BY SA 4.0 licence still applies.

\section{Other material for Integer Linear Programming}\label{integer-linear-programming}

Recall the problem on lemonade and lemon juice from Chapter
\ref{graphic}:

\textbf{Problem.} Say you are a vendor of lemonade and lemon juice. Each
unit of lemonade requires 1 lemon and 2 litres of water. Each unit of
lemon juice requires 3 lemons and 1 litre of water. Each unit of
lemonade gives a profit of \(\$ 3\). Each unit of lemon juice gives a
profit of \(\$ 2\). You have 6 lemons and 4 litres of water available.
How many units of lemonade and lemon juice should you make to maximize
profit?

Letting \(x\) denote the number of units of lemonade to be made and
letting \(y\) denote the number of units of lemon juice to be made, the
problem could be formulated as the following linear programming problem:

\[\begin{array}{rrcrll}
\max & 3x & + & 2y & \\
\text{s.t.} 
& x & + & 3y & \leq & 6 \\
& 2x & +&  y & \leq & 4 \\
& x &  & & \geq & 0 \\
& & & y & \geq & 0. \\
\end{array}\]

The problem has a unique optimal solution at
\(\begin{bmatrix} x \\ y\end{bmatrix} = \begin{bmatrix} 1.2 \\ 1.6\end{bmatrix}\)
for a profit of \(6.8\). But this solution requires us to make
fractional units of lemonade and lemon juice. What if we require the
number of units to be integers? In other words, we want to solve
\[\begin{array}{rrcrll}
\max & 3x & + & 2y & \\
\text{s.t.} 
& x & + & 3y & \leq & 6 \\
& 2x & +&  y & \leq & 4 \\
& x &  & & \geq & 0 \\
& & & y & \geq & 0 \\
& x &,& y & \in  & \mathbb{Z}. \\
\end{array}\] This problem is no longer a linear programming problem.
But rather, it is an integer linear programming problem.

A \textbf{mixed-integer linear programming problem} is a problem of
minimizing or maximizing a linear function subject to finitely many
linear constraints such that the number of variables are finite and at
least one of which is required to take on integer values.

If all the variables are required to take on integer values, the problem
is called a \textbf{pure integer linear programming problem} or simply
an \textbf{integer linear programming problem}. Normally, we assume the
problem data to be rational numbers to rule out some pathological cases.

Mixed-integer linear programming problems are in general difficult to
solve yet they are too important to ignore because they have a wide
range of applications (e.g.~transportation planning, crew scheduling,
circuit design, resource management etc.) Many solution methods for
these problems have been devised and some of them first solve the
\textbf{linear programming relaxation} of the original problem, which is
the problem obtained from the original problem by dropping all the
integer requirements on the variables.

\begin{example}{}{}
\protect\hypertarget{ex:ilp-ex}{}{\label{ex:ilp-ex}} Let (MP) denote the
following mixed-integer linear programming problem:
\[\begin{array}{rrcrcrlll}
\mbox{min} & x_1 &  &  & + & x_3  \\
\text{s.t.} & -x_1 & + &  x_2 & + &  x_3  & \geq & 1 \\
& -x_1 & - &  x_2 & + & 2x_3  & \geq & 0 \\
& -x_1 & + & 5x_2 & - &  x_3  & = & 3 \\
&  x_1 & , & x_2 & , & x_3 & \geq & 0 \\
&      &   &     &   & x_3  & \in & \mathbb{Z}. 
\end{array}\]

The linear programming relaxation of (MP) is: \[\begin{array}{rrcrcrlll}
\mbox{min} & x_1 &  &  & + & x_3  \\
\text{s.t.} & -x_1 & + &  x_2 & + &  x_3  & \geq & 1 \\
& -x_1 & - &  x_2 & + & 2x_3  & \geq & 0 \\
& -x_1 & + & 5x_2 & - &  x_3  & = & 3 \\
&  x_1 & , & x_2 & , & x_3 & \geq & 0.
\end{array}\]
\end{example}

Let (P1) denote the linear programming relaxation of (MP). Observe that
the optimal value of (P1) is a lower bound for the optimal value of (MP)
since the feasible region of (P1) contains all the feasible solutions to
(MP), thus making it possible to find a feasible solution to (P1) with
objective function value better than the optimal value of (MP). Hence,
if an optimal solution to the linear programming relaxation happens to
be a feasible solution to the original problem, then it is also an
optimal solution to the original problem. Otherwise, there is an integer
variable having a nonintegral value \(v\). What we then do is to create
two new subproblems as follows: one requiring the variable to be at most
the greatest integer less than \(v\), the other requiring the variable
to be at least the smallest integer greater than \(v\). This is the
basic idea behind the \textbf{branch-and-bound method}. We now
illustrate these ideas on (MP).

Solving the linear programming relaxation (P1), we find that
\(\vec{x}' = \begin{bmatrix}0\\ \frac{2}{3}\\ \frac{1}{3}\end{bmatrix}\)
is an optimal solution to (P1). Note that \(\mathbf{x}'\) is not a
feasible solution to (MP) because \(x'_3\) is not an integer. We now
create two subproblems (P2) and (P3) such that (P2) is obtained from
(P1) by adding the constraint \(x_3 \leq \lfloor x'_3\rfloor\) and (P3)
is obtained from (P1) by adding the constraint
\(x_3 \geq \lceil x'_3\rceil\). (For a number \(a\),
\(\lfloor a \rfloor\) denotes the greatest integer at most \(a\) and
\(\lceil a \rceil\) denotes the smallest integer at least \(a\).) Hence,
(P2) is the problem \[\begin{array}{rrcrcrlll}
\min & x_1 &  &  & + & x_3  \\
\text{s.t.} & -x_1 & + &  x_2 & + &  x_3  & \geq & 1 \\
 & -x_1 & - &  x_2 & + & 2x_3  & \geq & 0 \\
 & -x_1 & + & 5x_2 & - &  x_3  & = & 3 \\
 &      &   &      &   &  x_3  & \leq & 0 \\
 &  x_1 & , & x_2 & , & x_3 & \geq & 0,
\end{array}\] and (P3) is the problem \[\begin{array}{rrcrcrlll}
\min & x_1 &  &  & + & x_3  \\
\text{s.t.} & -x_1 & + &  x_2 & + &  x_3  & \geq & 1 \\
 & -x_1 & - &  x_2 & + & 2x_3  & \geq & 0 \\
 & -x_1 & + & 5x_2 & - &  x_3  & = & 3 \\
 &      &   &      &   &  x_3  & \geq & 1 \\
 &  x_1 & , & x_2 & , & x_3 & \geq & 0.
\end{array}\] Note that any feasible solution to (MP) must be a feasible
solution to either (P2) or (P3). Using the help of a solver, one sees
that (P2) is infeasible. The problem (P3) has an optimal solution at
\(\mathbf{x}^* = \begin{bmatrix}0\\ \frac{4}{5}\\ 1\end{bmatrix}\),
which is also feasible to (MP). Hence,
\(\mathbf{x}^* = \begin{bmatrix}0\\ \frac{4}{5}\\ 1\end{bmatrix}\) is an
optimal solution to (MP).

We now give a description of the method for a general mixed-integer
linear programming problem (MIP). Suppose that (MIP) is a minimization
problem and has \(n\) variables \(x_1,\ldots,x_n\). Let
\({\cal I} \subseteq \{1,\ldots,n\}\) denote the set of indices \(i\)
such that \(x_i\) is required to be an integer in (MIP).

\textbf{Branch-and-bound method}

\textbf{Input}: The problem (MIP).

\textbf{Steps}:

\begin{enumerate}
\def\labelenumi{\arabic{enumi}.}
\item
  Set \texttt{bestbound} \(:= \infty\), \(\vec{x}^*_{\text{best}}:= \)
  \texttt{N/A}, \texttt{activeproblems} \(:= \{ (LP) \}\) where \((LP)\)
  denotes the linear programming relaxation of (MIP).
\item
  If there is no problem in \texttt{activeproblems}, then stop; if
  \(\vec{x}^*_{\text{best}} \neq \) \texttt{N/A}, then
  \(\vec{x}^*_{\text{best}}\) is an optimal solution; otherwise, (MIP)
  has no optimal solution.
\item
  Select a problem \(P\) from \texttt{activeproblems} and remove it from
  \texttt{activeproblems}.
\item
  Solve \(P\).
\end{enumerate}

\begin{itemize}
\item
  If \(P\) is unbounded, then stop and conclude that (MIP) does not have
  an optimal solution.
\item
  If \(P\) is infeasible, go to step 2.
\item
  If \(P\) has an optimal solution \(\vec{x}^*\), then let \(z\) denote
  the objective function value of \(\vec{x}^*\).
\end{itemize}

\begin{enumerate}
\def\labelenumi{\arabic{enumi}.}
\setcounter{enumi}{4}
\item
  If \(z \geq \) \texttt{bestbound}, go to step 2.
\item
  If \(x^*_i\) is not an integer for some \(i \in {\cal I}\), then
  create two subproblems \(P_1\) and \(P_2\) such that \(P_1\) is the
  problem obtained from \(P\) by adding the constraint
  \(x_i \leq \lfloor x^*_i \rfloor\) and \(P_2\) is the problem obtained
  from \(P\) by adding the constraint \(x_i \geq \lceil x^*_i \rceil\).
  Add the problems \(P_1\) and \(P_2\) to \texttt{activeproblems} and go
  to step 2.
\item
  Set \(\vec{x}^*_{\text{best}} = \vec{x}^*\), \texttt{bestbound} \(=z\)
  and go to step 2.
\end{enumerate}

\textbf{Remarks.}

\begin{itemize}
\item
  Throughout the algorithm, \texttt{activeproblems} is a set of
  subproblems remained to be solved. Note that for each problem \(P\) in
  \texttt{activeproblems}, \(P\) is a linear programming problem and
  that any feasible solution to \(P\) satisfying the integrality
  requirements is a feasible solution to (MIP).
\item
  \(x^*_{\text{best}}\) is the feasible solution to (MIP) that has the
  best objective function value found so far and \texttt{bestbound} is
  its objective function value. It is often called an
  \textbf{incumbent}.
\item
  In practice, how a problem from \texttt{activeproblems} is selected in
  step 3 has an impact on the overall performance. However, there is no
  general rule for selection that guarantees good performance all the
  time.
\item
  In step 5, the problem \(P\) is discarded since it cannot contain any
  feasible solution to (MIP) having a better objective function value
  than \(x^*_{\text{best}}\).
\item
  If step 7 is reached, then \(x^*\) is a feasible solution to (MIP)
  having objective function value better than \texttt{bestbound}. So it
  becomes the current best solution.
\item
  It is possible for the algorithm to never terminate. Below is an
  example for which the algorithm will never stop:
  \[\begin{array}{rrcrcrcl}
  \min & x_1 \\
  \text{s.t.} & x_1 & + &  2x_2&  -&  2x_3&  =&  1 \\
  & x_1& , & x_2& , & x_3 & \geq & 0 \\
  & x_1& ,&  x_2& , & x_3 & \in & \mathbb{Z}.
  \end{array}
  \] However, it is easy to see that
  \(\vec{x}^* = \begin{bmatrix} 1 \\ 0 \\ 0\end{bmatrix}\) is an optimal
  solution because there is no feasible solution with \(x_1=0\).
\end{itemize}

One way to keep track of the progress of the computations is to set up a
progress chart with the following headings:

\begin{longtable}[]{@{}ccccccc@{}}
\toprule
\begin{minipage}[b]{0.08\columnwidth}\centering\strut
\textbf{Iter}\strut
\end{minipage} & \begin{minipage}[b]{0.08\columnwidth}\centering\strut
\textbf{solved}\strut
\end{minipage} & \begin{minipage}[b]{0.09\columnwidth}\centering\strut
\textbf{status}\strut
\end{minipage} & \begin{minipage}[b]{0.08\columnwidth}\centering\strut
\textbf{branching}\strut
\end{minipage} & \begin{minipage}[b]{0.18\columnwidth}\centering\strut
\texttt{activeproblems}\strut
\end{minipage} & \begin{minipage}[b]{0.09\columnwidth}\centering\strut
\(\mathbf{x}^*_{\text{best}}\)\strut
\end{minipage} & \begin{minipage}[b]{0.12\columnwidth}\centering\strut
\texttt{bestbound}\strut
\end{minipage}\tabularnewline
\midrule
\endhead
\end{longtable}

In a given iteration, the entry in the \textbf{solved} column denotes
the subproblem that has been solved with the result in the
\textbf{status} column. The \textbf{branching} column indicates the
subproblems created from the solved subproblem with an optimal solution
not feasible to (MIP). The entries in the remaining columns contain the
latest information in the given iteration. For the example (MP) above,
the chart could look like the following:

\begin{longtable}[]{@{}ccccccc@{}}
\toprule
\begin{minipage}[b]{0.07\columnwidth}\centering\strut
\textbf{Iter}\strut
\end{minipage} & \begin{minipage}[b]{0.08\columnwidth}\centering\strut
\textbf{solved}\strut
\end{minipage} & \begin{minipage}[b]{0.09\columnwidth}\centering\strut
\textbf{status}\strut
\end{minipage} & \begin{minipage}[b]{0.18\columnwidth}\centering\strut
\textbf{branching}\strut
\end{minipage} & \begin{minipage}[b]{0.15\columnwidth}\centering\strut
\texttt{activeproblems}\strut
\end{minipage} & \begin{minipage}[b]{0.07\columnwidth}\centering\strut
\(\mathbf{x}^*_{\text{best}}\)\strut
\end{minipage} & \begin{minipage}[b]{0.11\columnwidth}\centering\strut
\texttt{bestbound}\strut
\end{minipage}\tabularnewline
\midrule
\endhead
\begin{minipage}[t]{0.07\columnwidth}\centering\strut
1\strut
\end{minipage} & \begin{minipage}[t]{0.08\columnwidth}\centering\strut
(P1)\strut
\end{minipage} & \begin{minipage}[t]{0.09\columnwidth}\centering\strut
optimal
\(\mathbf{x}^*=\begin{bmatrix}0\\ \frac{2}{3}\\ \frac{1}{3}\end{bmatrix}\)\strut
\end{minipage} & \begin{minipage}[t]{0.18\columnwidth}\centering\strut
(P2): \(x_3 \leq 0\),\\ (P3): \(x_3 \geq 1\)\strut
\end{minipage} & \begin{minipage}[t]{0.15\columnwidth}\centering\strut
(P2), (P3)\strut
\end{minipage} & \begin{minipage}[t]{0.07\columnwidth}\centering\strut
N/A\strut
\end{minipage} & \begin{minipage}[t]{0.11\columnwidth}\centering\strut
\(\infty\)\strut
\end{minipage}\tabularnewline
\begin{minipage}[t]{0.07\columnwidth}\centering\strut
2\strut
\end{minipage} & \begin{minipage}[t]{0.08\columnwidth}\centering\strut
(P2)\strut
\end{minipage} & \begin{minipage}[t]{0.09\columnwidth}\centering\strut
infeasible\strut
\end{minipage} & \begin{minipage}[t]{0.18\columnwidth}\centering\strut
---\strut
\end{minipage} & \begin{minipage}[t]{0.15\columnwidth}\centering\strut
(P3)\strut
\end{minipage} & \begin{minipage}[t]{0.07\columnwidth}\centering\strut
N/A\strut
\end{minipage} & \begin{minipage}[t]{0.11\columnwidth}\centering\strut
\(\infty\)\strut
\end{minipage}\tabularnewline
\begin{minipage}[t]{0.07\columnwidth}\centering\strut
3\strut
\end{minipage} & \begin{minipage}[t]{0.08\columnwidth}\centering\strut
(P3)\strut
\end{minipage} & \begin{minipage}[t]{0.09\columnwidth}\centering\strut
optimal
\(\mathbf{x}^*=\begin{bmatrix}0\\ \frac{4}{5}\\ 1\end{bmatrix}\)\strut
\end{minipage} & \begin{minipage}[t]{0.18\columnwidth}\centering\strut
---\strut
\end{minipage} & \begin{minipage}[t]{0.15\columnwidth}\centering\strut
---\strut
\end{minipage} & \begin{minipage}[t]{0.07\columnwidth}\centering\strut
\(\begin{bmatrix}0\\ \frac{4}{5}\\ 1\end{bmatrix}\)\strut
\end{minipage} & \begin{minipage}[t]{0.11\columnwidth}\centering\strut
\(1\)\strut
\end{minipage}\tabularnewline
\bottomrule
\end{longtable}

\subsection*{Exercises}\label{exercises-10}
\addcontentsline{toc}{section}{Exercises}

\begin{enumerate}
\def\labelenumi{\arabic{enumi}.}
\item
  Suppose that (MP) in Example \ref{ex:ilp-ex} above has \(x_2\)
  required to be an integer as well. Continue with the computations and
  determine an optimal solution to the modified problem.
\item
  With the help of a solver, determine the optimal value of
  \[\begin{array}{rrcrll}
  \max & 3x & + & 2y & \\
  \text{s.t.} 
  & x & + & 3y & \leq & 6 \\
  & 2x & +&  y & \leq & 4 \\
  & x & ,& y & \geq & 0 \\
  & x &,& y & \in  & \Z. \\
  \end{array}\]
\item
  Let \(\mm{A} \in \Q^{m\times n}\) and \(\vec{b} \in \Q^m\). Let
  \(S\) denote the system
  \begin{align*}
    \mm{A} \vec{x} & \geq \vec{b}\\
    \vec{x} & \in \Z^n
  \end{align*}

  \begin{enumerate}
  \def\labelenumii{\alph{enumii}.}
  \item
    Suppose that \(\vec{d} \in \Q^m\) satisfies
    \(\vec{d} \geq \vec{0}\) and \(\vec{d}^\T\mm{A} \in \Z^n\). Prove
    that every \(\vec{x}\) satisfying \(S\) also satisfies
    \(\vec{d}^T\mm{A} \vec{x} \geq \lceil \vec{d}^\T\vec{b}\rceil\).
    (This inequality is known as a \textbf{Chvátal-Gomory cutting
    plane.})
  \item
    Suppose that
    \(\mm{A} = \begin{bmatrix} 2 & 3 \\ 5 & 3 \\ 7 & 6 \end{bmatrix}\)
    and \(\vec{b} = \begin{bmatrix} 2 \\ 1 \\ 8\end{bmatrix}\). Show
    that every \(\vec{x}\) satisfying \(S\) also satisfies
    \(x_1 + x_2 \geq 2\).
  \end{enumerate}
\end{enumerate}

\subsection*{Solutions}\label{solutions-10}
\addcontentsline{toc}{section}{Solutions}

\begin{enumerate}
\def\labelenumi{\arabic{enumi}.}
\item
  An optimal solution to the modified problem is given by
  \(x^* = \begin{bmatrix} 1\\1\\1 \end{bmatrix}\).
\item
  An optimal solution is
  \(\begin{bmatrix} x\\y \end{bmatrix} = \begin{bmatrix} 2 \\ 0\end{bmatrix}\).
  Thus, the optimal value is \(6\).
\item
  \begin{enumerate}
  \def\labelenumii{\alph{enumii}.}
  \item
    Since \(\vec{d} \geq \vec{0}\) and \(\mm{A}\vec{x} \geq \vec{b}\),
    we have \(\vec{d}^T\mm{A} \vec{x} \geq \vec{d}^\T\vec{b}\). If
    \(\vec{d}^\T\vec{b}\) is an integer, the result follows immediately.
    Otherwise, note that \(\vec{d}^\T\mm{A} \in \Z^n\) and
    \(\vec{x}\in \Z^n\) imply that \(\vec{d}^T\mm{A} \vec{x}\) is an
    integer. Thus, \(\vec{d}^T\mm{A} \vec{x}\) must be greater than or
    equal to the least integer greater than \(\vec{d}^\T\vec{b}\).
  \item
    Take
    \(\vec{d} = \begin{bmatrix} \frac{1}{9} \\ 0 \\ \frac{1}{9} \end{bmatrix}\)
    and apply the result in the previous part.
  \end{enumerate}
\end{enumerate}


