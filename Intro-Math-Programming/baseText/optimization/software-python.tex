\chapter{Software - Python}
\todoChapter{ {\color{gray}10\% complete. Goal 80\% completion date: August 20}\\
Notes: }
\begin{outcome}
\begin{itemize}
\item Install and get python up and running in some form
\item Introduce basic python skills that will be helpful
\end{itemize}
\end{outcome}

\begin{resource}
\begin{itemize}
\item\href{https://open.umn.edu/opentextbooks/textbooks/a-byte-of-python}{A Byte of Python}
\item\href{https://github.com/swaroopch/byte-of-python}{Github - Byte of Python} (CC-BY-SA)
\end{itemize}
\end{resource}

\subimport{./foundationsAppliedMathematicsLabs/Appendices/Installation/}{Installation} %Chapter
\subimport{./foundationsAppliedMathematicsLabs/Appendices/NumpyVisualGuide/}{NumpyVisualGuide} %Chapter
\subimport{./foundationsAppliedMathematicsLabs/Appendices/MatplotlibCustomization/}{MatplotlibCustomization} %Chapter

\section{Networkx - A Python Graph Algorithms Package}


\section{PuLP - An Optimization Modeling Tool for Python}
\begin{outcome}
\begin{itemize}
\item Install and import PuLP
\item Run basic first PuLP model
\item Run "advanced" PuLP model using the algebraic modeling approach and importing data.
\item Explore PuLP objects and possibilities
\item Solve a Multi-Objective problem
\end{itemize}
\end{outcome}


\begin{resource}
\begin{itemize}
\item \href{https://coin-or.github.io/pulp/}{Documentation}
\item \href{https://pypi.org/project/PuLP/}{PyPi installation}
\item \href{https://github.com/coin-or/pulp/tree/master/examples}{Examples}
\item \href{https://benalexkeen.com/linear-programming-with-python-and-pulp-part-1/}{Blog with tutorial}
\end{itemize}
\end{resource}

PuLP is an optimizaiton modeling language that is written for Python.  It is free and open source.  Yay!   See Section ?? for a discussion of other options for implementing your optimization problem.
PuLP is convenient for it's simple syntax and easy installation.   
 
Key benefits of using an algebraic modeling language like PuLP over Excel
\begin{itemize}
\item Easily readable models
\item Precompute parameters within Python
\item Reuse of common optimization models without recreating the equations
\end{itemize}

We will follow the introduction to pulp Jupyter Notebook Tutorial and the following application with a cleaner implementation.



%    \hypertarget{pulp-tutorial}{%
%\subsection{Pulp Tutorial}\label{pulp-tutorial}}

    \hypertarget{installation}{%
\subsection{Installation}\label{installation}}

Open a Jupyter notebook. In one of the cells, run the following command,
based on which system you are running. It will take a minute to load and
download the package.

    \begin{tcolorbox}[breakable, size=fbox, boxrule=1pt, pad at break*=1mm,colback=cellbackground, colframe=cellborder]
\prompt{In}{incolor}{ }{\boxspacing}
\begin{Verbatim}[commandchars=\\\{\}]
\PY{c+c1}{\PYZsh{}\PYZsh{} Install pulp (on windows)}
\PY{o}{!}pip install pulp
\end{Verbatim}
\end{tcolorbox}

    \begin{tcolorbox}[breakable, size=fbox, boxrule=1pt, pad at break*=1mm,colback=cellbackground, colframe=cellborder]
\prompt{In}{incolor}{ }{\boxspacing}
\begin{Verbatim}[commandchars=\\\{\}]
\PY{c+c1}{\PYZsh{} on a mac}
\PY{n}{pip} \PY{n}{install} \PY{n}{pulp}
\end{Verbatim}
\end{tcolorbox}

    \begin{tcolorbox}[breakable, size=fbox, boxrule=1pt, pad at break*=1mm,colback=cellbackground, colframe=cellborder]
\prompt{In}{incolor}{ }{\boxspacing}
\begin{Verbatim}[commandchars=\\\{\}]
\PY{c+c1}{\PYZsh{} on the VT ARC servers}
\PY{k+kn}{import} \PY{n+nn}{sys}
\PY{o}{!}\PY{o}{\PYZob{}}sys.executable\PY{o}{\PYZcb{}} \PYZhy{}m pip install pulp
\end{Verbatim}
\end{tcolorbox}

    \#\#\# Installation (Continued) Now restart the kernel of your notebook
(find the tab labeled Kernel in your Jupyter notebook, and in the drop
down, select restart).

    \hypertarget{example-problem}{%
\subsection{Example Problem}\label{example-problem}}

\hypertarget{product-mix-problem}{%
\subsubsection{Product Mix Problem}\label{product-mix-problem}}

\begin{align*}
  & \text{maximize }   &   Z=3&X_{1}+2X_{2}         & \text{(Objective function)} &\quad(1.1)\\[1ex]
  & \text{subject to } & \, 10&X_{1}+5X_{2} \le 300 & \text{(Constraint 1)}       &\quad(1.2)\\[1ex]
  &                    & \,  4&X_{1}+4X_{2} \le 160 & \text{(Constraint 2)}       &\quad(1.3)\\[1ex]  
  &                    & \,  2&X_{1}+6X_{2} \le 180 & \text{(Constraint 3)}       &\quad(1.4)\\[1ex] 
  & \text{and}         & \,   &X_{1},X_{2} \ge 0    & \text{(Non-negative)}       &\quad(1.5)\\[1ex] 
\end{align*}

    \hypertarget{optimization-with-pulp}{%
\paragraph{Optimization with PuLP}\label{optimization-with-pulp}}

    \begin{tcolorbox}[breakable, size=fbox, boxrule=1pt, pad at break*=1mm,colback=cellbackground, colframe=cellborder]
\prompt{In}{incolor}{1}{\boxspacing}
\begin{Verbatim}[commandchars=\\\{\}]
\PY{k+kn}{from} \PY{n+nn}{pulp} \PY{k+kn}{import} \PY{o}{*}

\PY{c+c1}{\PYZsh{} Define problem}
\PY{n}{prob} \PY{o}{=} \PY{n}{LpProblem}\PY{p}{(}\PY{n}{name}\PY{o}{=}\PY{l+s+s1}{\PYZsq{}}\PY{l+s+s1}{Product\PYZus{}Mix\PYZus{}Problem}\PY{l+s+s1}{\PYZsq{}}\PY{p}{,} \PY{n}{sense}\PY{o}{=}\PY{n}{LpMaximize}\PY{p}{)}

\PY{c+c1}{\PYZsh{} Create decision variables and non\PYZhy{}negative constraint}
\PY{n}{x1} \PY{o}{=} \PY{n}{LpVariable}\PY{p}{(}\PY{n}{name}\PY{o}{=}\PY{l+s+s1}{\PYZsq{}}\PY{l+s+s1}{X1}\PY{l+s+s1}{\PYZsq{}}\PY{p}{,} \PY{n}{lowBound}\PY{o}{=}\PY{l+m+mi}{0}\PY{p}{,} \PY{n}{upBound}\PY{o}{=}\PY{k+kc}{None}\PY{p}{,} \PY{n}{cat}\PY{o}{=}\PY{l+s+s1}{\PYZsq{}}\PY{l+s+s1}{Continuous}\PY{l+s+s1}{\PYZsq{}}\PY{p}{)}
\PY{n}{x2} \PY{o}{=} \PY{n}{LpVariable}\PY{p}{(}\PY{n}{name}\PY{o}{=}\PY{l+s+s1}{\PYZsq{}}\PY{l+s+s1}{X2}\PY{l+s+s1}{\PYZsq{}}\PY{p}{,} \PY{n}{lowBound}\PY{o}{=}\PY{l+m+mi}{0}\PY{p}{,} \PY{n}{upBound}\PY{o}{=}\PY{k+kc}{None}\PY{p}{,} \PY{n}{cat}\PY{o}{=}\PY{l+s+s1}{\PYZsq{}}\PY{l+s+s1}{Continuous}\PY{l+s+s1}{\PYZsq{}}\PY{p}{)}

\PY{c+c1}{\PYZsh{} Set objective function}
\PY{n}{prob} \PY{o}{+}\PY{o}{=} \PY{l+m+mi}{3}\PY{o}{*}\PY{n}{x1} \PY{o}{+} \PY{l+m+mi}{2}\PY{o}{*}\PY{n}{x2}

\PY{c+c1}{\PYZsh{} Set constraints}
\PY{n}{prob} \PY{o}{+}\PY{o}{=} \PY{l+m+mi}{10}\PY{o}{*}\PY{n}{x1} \PY{o}{+} \PY{l+m+mi}{5}\PY{o}{*}\PY{n}{x2} \PY{o}{\PYZlt{}}\PY{o}{=} \PY{l+m+mi}{300}
\PY{n}{prob} \PY{o}{+}\PY{o}{=} \PY{l+m+mi}{4}\PY{o}{*}\PY{n}{x1} \PY{o}{+} \PY{l+m+mi}{4}\PY{o}{*}\PY{n}{x2} \PY{o}{\PYZlt{}}\PY{o}{=} \PY{l+m+mi}{160}
\PY{n}{prob} \PY{o}{+}\PY{o}{=} \PY{l+m+mi}{2}\PY{o}{*}\PY{n}{x1} \PY{o}{+} \PY{l+m+mi}{6}\PY{o}{*}\PY{n}{x2} \PY{o}{\PYZlt{}}\PY{o}{=} \PY{l+m+mi}{180}

\PY{c+c1}{\PYZsh{} Solving problem}
\PY{n}{prob}\PY{o}{.}\PY{n}{solve}\PY{p}{(}\PY{p}{)}
\PY{n+nb}{print}\PY{p}{(}\PY{l+s+s1}{\PYZsq{}}\PY{l+s+s1}{Status}\PY{l+s+s1}{\PYZsq{}}\PY{p}{,} \PY{n}{LpStatus}\PY{p}{[}\PY{n}{prob}\PY{o}{.}\PY{n}{status}\PY{p}{]}\PY{p}{)}
\end{Verbatim}
\end{tcolorbox}

    \begin{Verbatim}[commandchars=\\\{\}]
Status Optimal
    \end{Verbatim}

    \begin{tcolorbox}[breakable, size=fbox, boxrule=1pt, pad at break*=1mm,colback=cellbackground, colframe=cellborder]
\prompt{In}{incolor}{2}{\boxspacing}
\begin{Verbatim}[commandchars=\\\{\}]
\PY{n+nb}{print}\PY{p}{(}\PY{l+s+s2}{\PYZdq{}}\PY{l+s+s2}{Status:}\PY{l+s+s2}{\PYZdq{}}\PY{p}{,} \PY{n}{LpStatus}\PY{p}{[}\PY{n}{prob}\PY{o}{.}\PY{n}{status}\PY{p}{]}\PY{p}{)}
\PY{n+nb}{print}\PY{p}{(}\PY{l+s+s2}{\PYZdq{}}\PY{l+s+s2}{Objective value: }\PY{l+s+s2}{\PYZdq{}}\PY{p}{,} \PY{n}{prob}\PY{o}{.}\PY{n}{objective}\PY{o}{.}\PY{n}{value}\PY{p}{(}\PY{p}{)}\PY{p}{)}

\PY{k}{for} \PY{n}{v} \PY{o+ow}{in} \PY{n}{prob}\PY{o}{.}\PY{n}{variables}\PY{p}{(}\PY{p}{)}\PY{p}{:}
    \PY{n+nb}{print}\PY{p}{(}\PY{n}{v}\PY{o}{.}\PY{n}{name}\PY{p}{,}\PY{l+s+s1}{\PYZsq{}}\PY{l+s+s1}{: }\PY{l+s+s1}{\PYZsq{}}\PY{p}{,} \PY{n}{v}\PY{o}{.}\PY{n}{value}\PY{p}{(}\PY{p}{)}\PY{p}{)}
\end{Verbatim}
\end{tcolorbox}

    \begin{Verbatim}[commandchars=\\\{\}]
Status: Optimal
Objective value:  100.0
X1 :  20.0
X2 :  20.0
    \end{Verbatim}

    \hypertarget{things-we-can-do}{%
\subsection{Things we can do}\label{things-we-can-do}}

    \begin{tcolorbox}[breakable, size=fbox, boxrule=1pt, pad at break*=1mm,colback=cellbackground, colframe=cellborder]
\prompt{In}{incolor}{3}{\boxspacing}
\begin{Verbatim}[commandchars=\\\{\}]
\PY{c+c1}{\PYZsh{} print the problem}
\PY{n}{prob}
\end{Verbatim}
\end{tcolorbox}

            \begin{tcolorbox}[breakable, size=fbox, boxrule=.5pt, pad at break*=1mm, opacityfill=0]
\prompt{Out}{outcolor}{3}{\boxspacing}
\begin{Verbatim}[commandchars=\\\{\}]
Product\_Mix\_Problem:
MAXIMIZE
3*X1 + 2*X2 + 0
SUBJECT TO
\_C1: 10 X1 + 5 X2 <= 300

\_C2: 4 X1 + 4 X2 <= 160

\_C3: 2 X1 + 6 X2 <= 180

VARIABLES
X1 Continuous
X2 Continuous
\end{Verbatim}
\end{tcolorbox}
        
    \begin{tcolorbox}[breakable, size=fbox, boxrule=1pt, pad at break*=1mm,colback=cellbackground, colframe=cellborder]
\prompt{In}{incolor}{4}{\boxspacing}
\begin{Verbatim}[commandchars=\\\{\}]
\PY{c+c1}{\PYZsh{} get the objective function}
\PY{n}{prob}\PY{o}{.}\PY{n}{objective}\PY{o}{.}\PY{n}{value}\PY{p}{(}\PY{p}{)}
\end{Verbatim}
\end{tcolorbox}

            \begin{tcolorbox}[breakable, size=fbox, boxrule=.5pt, pad at break*=1mm, opacityfill=0]
\prompt{Out}{outcolor}{4}{\boxspacing}
\begin{Verbatim}[commandchars=\\\{\}]
100.0
\end{Verbatim}
\end{tcolorbox}
        
    \begin{tcolorbox}[breakable, size=fbox, boxrule=1pt, pad at break*=1mm,colback=cellbackground, colframe=cellborder]
\prompt{In}{incolor}{5}{\boxspacing}
\begin{Verbatim}[commandchars=\\\{\}]
\PY{c+c1}{\PYZsh{} get list of the variables}
\PY{n}{prob}\PY{o}{.}\PY{n}{variables}\PY{p}{(}\PY{p}{)}
\end{Verbatim}
\end{tcolorbox}

            \begin{tcolorbox}[breakable, size=fbox, boxrule=.5pt, pad at break*=1mm, opacityfill=0]
\prompt{Out}{outcolor}{5}{\boxspacing}
\begin{Verbatim}[commandchars=\\\{\}]
[X1, X2]
\end{Verbatim}
\end{tcolorbox}
        
    \begin{tcolorbox}[breakable, size=fbox, boxrule=1pt, pad at break*=1mm,colback=cellbackground, colframe=cellborder]
\prompt{In}{incolor}{6}{\boxspacing}
\begin{Verbatim}[commandchars=\\\{\}]
\PY{k}{for} \PY{n}{v} \PY{o+ow}{in} \PY{n}{prob}\PY{o}{.}\PY{n}{variables}\PY{p}{(}\PY{p}{)}\PY{p}{:}
    \PY{n+nb}{print}\PY{p}{(}\PY{l+s+sa}{f}\PY{l+s+s1}{\PYZsq{}}\PY{l+s+si}{\PYZob{}}\PY{n}{v}\PY{l+s+si}{\PYZcb{}}\PY{l+s+s1}{: }\PY{l+s+si}{\PYZob{}}\PY{n}{v}\PY{o}{.}\PY{n}{varValue}\PY{l+s+si}{\PYZcb{}}\PY{l+s+s1}{\PYZsq{}}\PY{p}{)}
\end{Verbatim}
\end{tcolorbox}

    \begin{Verbatim}[commandchars=\\\{\}]
X1: 20.0
X2: 20.0
    \end{Verbatim}

    \hypertarget{exploring-the-variables}{%
\subsubsection{Exploring the variables}\label{exploring-the-variables}}

    \begin{tcolorbox}[breakable, size=fbox, boxrule=1pt, pad at break*=1mm,colback=cellbackground, colframe=cellborder]
\prompt{In}{incolor}{7}{\boxspacing}
\begin{Verbatim}[commandchars=\\\{\}]
\PY{n}{v} \PY{o}{=} \PY{n}{prob}\PY{o}{.}\PY{n}{variables}\PY{p}{(}\PY{p}{)}\PY{p}{[}\PY{l+m+mi}{0}\PY{p}{]}
\end{Verbatim}
\end{tcolorbox}

    \begin{tcolorbox}[breakable, size=fbox, boxrule=1pt, pad at break*=1mm,colback=cellbackground, colframe=cellborder]
\prompt{In}{incolor}{9}{\boxspacing}
\begin{Verbatim}[commandchars=\\\{\}]
\PY{n}{v}\PY{o}{.}\PY{n}{name}
\end{Verbatim}
\end{tcolorbox}

            \begin{tcolorbox}[breakable, size=fbox, boxrule=.5pt, pad at break*=1mm, opacityfill=0]
\prompt{Out}{outcolor}{9}{\boxspacing}
\begin{Verbatim}[commandchars=\\\{\}]
'X1'
\end{Verbatim}
\end{tcolorbox}
        
    \begin{tcolorbox}[breakable, size=fbox, boxrule=1pt, pad at break*=1mm,colback=cellbackground, colframe=cellborder]
\prompt{In}{incolor}{10}{\boxspacing}
\begin{Verbatim}[commandchars=\\\{\}]
\PY{n}{v}\PY{o}{.}\PY{n}{value}\PY{p}{(}\PY{p}{)}
\end{Verbatim}
\end{tcolorbox}

            \begin{tcolorbox}[breakable, size=fbox, boxrule=.5pt, pad at break*=1mm, opacityfill=0]
\prompt{Out}{outcolor}{10}{\boxspacing}
\begin{Verbatim}[commandchars=\\\{\}]
20.0
\end{Verbatim}
\end{tcolorbox}
        
    \begin{tcolorbox}[breakable, size=fbox, boxrule=1pt, pad at break*=1mm,colback=cellbackground, colframe=cellborder]
\prompt{In}{incolor}{11}{\boxspacing}
\begin{Verbatim}[commandchars=\\\{\}]
\PY{n}{v}\PY{o}{.}\PY{n}{varValue}
\end{Verbatim}
\end{tcolorbox}

            \begin{tcolorbox}[breakable, size=fbox, boxrule=.5pt, pad at break*=1mm, opacityfill=0]
\prompt{Out}{outcolor}{11}{\boxspacing}
\begin{Verbatim}[commandchars=\\\{\}]
20.0
\end{Verbatim}
\end{tcolorbox}
        
    \hypertarget{other-things-you-can-do}{%
\subsubsection{Other things you can do}\label{other-things-you-can-do}}

    \begin{tcolorbox}[breakable, size=fbox, boxrule=1pt, pad at break*=1mm,colback=cellbackground, colframe=cellborder]
\prompt{In}{incolor}{12}{\boxspacing}
\begin{Verbatim}[commandchars=\\\{\}]
\PY{c+c1}{\PYZsh{} get list of the constraints}
\PY{n}{prob}\PY{o}{.}\PY{n}{constraints}
\end{Verbatim}
\end{tcolorbox}

            \begin{tcolorbox}[breakable, size=fbox, boxrule=.5pt, pad at break*=1mm, opacityfill=0]
\prompt{Out}{outcolor}{12}{\boxspacing}
\begin{Verbatim}[commandchars=\\\{\}]
OrderedDict([('\_C1', 10*X1 + 5*X2 + -300 <= 0),
             ('\_C2', 4*X1 + 4*X2 + -160 <= 0),
             ('\_C3', 2*X1 + 6*X2 + -180 <= 0)])
\end{Verbatim}
\end{tcolorbox}
        
    \begin{tcolorbox}[breakable, size=fbox, boxrule=1pt, pad at break*=1mm,colback=cellbackground, colframe=cellborder]
\prompt{In}{incolor}{13}{\boxspacing}
\begin{Verbatim}[commandchars=\\\{\}]
\PY{n}{prob}\PY{o}{.}\PY{n}{to\PYZus{}dict}\PY{p}{(}\PY{p}{)}
\end{Verbatim}
\end{tcolorbox}

            \begin{tcolorbox}[breakable, size=fbox, boxrule=.5pt, pad at break*=1mm, opacityfill=0]
\prompt{Out}{outcolor}{13}{\boxspacing}
\begin{Verbatim}[commandchars=\\\{\}]
\{'objective': \{'name': 'OBJ',
  'coefficients': [\{'name': 'X1', 'value': 3\}, \{'name': 'X2', 'value': 2\}]\},
 'constraints': [\{'sense': -1,
   'pi': 0.2,
   'constant': -300,
   'name': None,
   'coefficients': [\{'name': 'X1', 'value': 10\}, \{'name': 'X2', 'value': 5\}]\},
  \{'sense': -1,
   'pi': 0.25,
   'constant': -160,
   'name': None,
   'coefficients': [\{'name': 'X1', 'value': 4\}, \{'name': 'X2', 'value': 4\}]\},
  \{'sense': -1,
   'pi': -0.0,
   'constant': -180,
   'name': None,
   'coefficients': [\{'name': 'X1', 'value': 2\}, \{'name': 'X2', 'value': 6\}]\}],
 'variables': [\{'lowBound': 0,
   'upBound': None,
   'cat': 'Continuous',
   'varValue': 20.0,
   'dj': -0.0,
   'name': 'X1'\},
  \{'lowBound': 0,
   'upBound': None,
   'cat': 'Continuous',
   'varValue': 20.0,
   'dj': -0.0,
   'name': 'X2'\}],
 'parameters': \{'name': 'Product\_Mix\_Problem',
  'sense': -1,
  'status': 1,
  'sol\_status': 1\},
 'sos1': [],
 'sos2': []\}
\end{Verbatim}
\end{tcolorbox}
        
    \begin{tcolorbox}[breakable, size=fbox, boxrule=1pt, pad at break*=1mm,colback=cellbackground, colframe=cellborder]
\prompt{In}{incolor}{15}{\boxspacing}
\begin{Verbatim}[commandchars=\\\{\}]
\PY{c+c1}{\PYZsh{} Store problem information in a json}
\PY{n}{prob}\PY{o}{.}\PY{n}{to\PYZus{}json}\PY{p}{(}\PY{l+s+s1}{\PYZsq{}}\PY{l+s+s1}{Product\PYZus{}Mix\PYZus{}Problem.json}\PY{l+s+s1}{\PYZsq{}}\PY{p}{)}
\end{Verbatim}
\end{tcolorbox}

    \hypertarget{common-issue}{%
\subsection{Common issue}\label{common-issue}}

If you forget the \textless=, ==, or \textgreater= when writing a
constraint, you will silently overwrite the objective function instead
of adding a constraint!






\hypertarget{transportation-problem}{%
\subsubsection{Transportation Problem}\label{transportation-problem}}

Transport programming is a special form of linear programming, and in
general, the objective function is cost minimization. The formula form
and applicable variables of the Transport Planning Act are as follows.
When supply and demand match, the constraint becomes an equation, but
when supply and demand do not match, the constraint becomes an
inequality.

Sets: - J = set of demand nodes - I = set of supply nodes

Parameters: 
\begin{itemize}
\item  $D_j$: Demand at node $j$ 
\item  $S_i$: Supply from node i 
\item  $c_{ij}$:
cost per unit to send supply i to demand j
\end{itemize}

Variables: 
\begin{itemize}
\item  X\_ij: Transport volume from supply \(i\) to demand \(j\)
(units)
\end{itemize}

\begin{itemize}
\item
  Objective function: \[\min \sum_{i=1}^n\sum_{j=1}^mc_{ij}x_{ij}\]
\item
  Constraints:
  \[\sum_{i=1}^nx_{ij}=S_i\]
  \[\sum_{i=1}^mx_{ij}=D_j\]
 \[ x_{ij}\geq 0 \text{ for } i \in I, j \in J\]
\end{itemize}

    \hypertarget{optimization-with-pulp}{%
\subsubsection{Optimization with PuLP}\label{optimization-with-pulp}}

Here we do a very basic implementation of the problem

    \begin{tcolorbox}[breakable, size=fbox, boxrule=1pt, pad at break*=1mm,colback=cellbackground, colframe=cellborder]
\prompt{In}{incolor}{1}{\boxspacing}
\begin{Verbatim}[commandchars=\\\{\}]
\PY{k+kn}{from} \PY{n+nn}{pulp} \PY{k+kn}{import} \PY{o}{*}

\PY{n}{prob} \PY{o}{=} \PY{n}{LpProblem}\PY{p}{(}\PY{l+s+s1}{\PYZsq{}}\PY{l+s+s1}{Transportation\PYZus{}Problem}\PY{l+s+s1}{\PYZsq{}}\PY{p}{,} \PY{n}{LpMinimize}\PY{p}{)}

\PY{n}{x11} \PY{o}{=} \PY{n}{LpVariable}\PY{p}{(}\PY{l+s+s1}{\PYZsq{}}\PY{l+s+s1}{X11}\PY{l+s+s1}{\PYZsq{}}\PY{p}{,} \PY{n}{lowBound}\PY{o}{=}\PY{l+m+mi}{0}\PY{p}{)}
\PY{n}{x12} \PY{o}{=} \PY{n}{LpVariable}\PY{p}{(}\PY{l+s+s1}{\PYZsq{}}\PY{l+s+s1}{X12}\PY{l+s+s1}{\PYZsq{}}\PY{p}{,} \PY{n}{lowBound}\PY{o}{=}\PY{l+m+mi}{0}\PY{p}{)}
\PY{n}{x13} \PY{o}{=} \PY{n}{LpVariable}\PY{p}{(}\PY{l+s+s1}{\PYZsq{}}\PY{l+s+s1}{X13}\PY{l+s+s1}{\PYZsq{}}\PY{p}{,} \PY{n}{lowBound}\PY{o}{=}\PY{l+m+mi}{0}\PY{p}{)}
\PY{n}{x14} \PY{o}{=} \PY{n}{LpVariable}\PY{p}{(}\PY{l+s+s1}{\PYZsq{}}\PY{l+s+s1}{X14}\PY{l+s+s1}{\PYZsq{}}\PY{p}{,} \PY{n}{lowBound}\PY{o}{=}\PY{l+m+mi}{0}\PY{p}{)}
\PY{n}{x21} \PY{o}{=} \PY{n}{LpVariable}\PY{p}{(}\PY{l+s+s1}{\PYZsq{}}\PY{l+s+s1}{X21}\PY{l+s+s1}{\PYZsq{}}\PY{p}{,} \PY{n}{lowBound}\PY{o}{=}\PY{l+m+mi}{0}\PY{p}{)}
\PY{n}{x22} \PY{o}{=} \PY{n}{LpVariable}\PY{p}{(}\PY{l+s+s1}{\PYZsq{}}\PY{l+s+s1}{X22}\PY{l+s+s1}{\PYZsq{}}\PY{p}{,} \PY{n}{lowBound}\PY{o}{=}\PY{l+m+mi}{0}\PY{p}{)}
\PY{n}{x23} \PY{o}{=} \PY{n}{LpVariable}\PY{p}{(}\PY{l+s+s1}{\PYZsq{}}\PY{l+s+s1}{X23}\PY{l+s+s1}{\PYZsq{}}\PY{p}{,} \PY{n}{lowBound}\PY{o}{=}\PY{l+m+mi}{0}\PY{p}{)}
\PY{n}{x24} \PY{o}{=} \PY{n}{LpVariable}\PY{p}{(}\PY{l+s+s1}{\PYZsq{}}\PY{l+s+s1}{X24}\PY{l+s+s1}{\PYZsq{}}\PY{p}{,} \PY{n}{lowBound}\PY{o}{=}\PY{l+m+mi}{0}\PY{p}{)}
\PY{n}{x31} \PY{o}{=} \PY{n}{LpVariable}\PY{p}{(}\PY{l+s+s1}{\PYZsq{}}\PY{l+s+s1}{X31}\PY{l+s+s1}{\PYZsq{}}\PY{p}{,} \PY{n}{lowBound}\PY{o}{=}\PY{l+m+mi}{0}\PY{p}{)}
\PY{n}{x32} \PY{o}{=} \PY{n}{LpVariable}\PY{p}{(}\PY{l+s+s1}{\PYZsq{}}\PY{l+s+s1}{X32}\PY{l+s+s1}{\PYZsq{}}\PY{p}{,} \PY{n}{lowBound}\PY{o}{=}\PY{l+m+mi}{0}\PY{p}{)}
\PY{n}{x33} \PY{o}{=} \PY{n}{LpVariable}\PY{p}{(}\PY{l+s+s1}{\PYZsq{}}\PY{l+s+s1}{X33}\PY{l+s+s1}{\PYZsq{}}\PY{p}{,} \PY{n}{lowBound}\PY{o}{=}\PY{l+m+mi}{0}\PY{p}{)}
\PY{n}{x34} \PY{o}{=} \PY{n}{LpVariable}\PY{p}{(}\PY{l+s+s1}{\PYZsq{}}\PY{l+s+s1}{X34}\PY{l+s+s1}{\PYZsq{}}\PY{p}{,} \PY{n}{lowBound}\PY{o}{=}\PY{l+m+mi}{0}\PY{p}{)}

\PY{n}{prob} \PY{o}{+}\PY{o}{=} \PY{l+m+mi}{4}\PY{o}{*}\PY{n}{x11} \PY{o}{+} \PY{l+m+mi}{5}\PY{o}{*}\PY{n}{x12} \PY{o}{+} \PY{l+m+mi}{6}\PY{o}{*}\PY{n}{x13} \PY{o}{+} \PY{l+m+mi}{8}\PY{o}{*}\PY{n}{x14} \PY{o}{+} \PY{l+m+mi}{4}\PY{o}{*}\PY{n}{x21} \PY{o}{+} \PY{l+m+mi}{7}\PY{o}{*}\PY{n}{x22} \PY{o}{+} \PY{l+m+mi}{9}\PY{o}{*}\PY{n}{x23} \PY{o}{+} \PY{l+m+mi}{2}\PY{o}{*}\PY{n}{x24} \PY{o}{+} \PY{l+m+mi}{5}\PY{o}{*}\PY{n}{x31} \PY{o}{+} \PY{l+m+mi}{8}\PY{o}{*}\PY{n}{x32} \PY{o}{+} \PY{l+m+mi}{7}\PY{o}{*}\PY{n}{x33} \PY{o}{+} \PY{l+m+mi}{6}\PY{o}{*}\PY{n}{x34}

\PY{n}{prob} \PY{o}{+}\PY{o}{=} \PY{n}{x11} \PY{o}{+} \PY{n}{x12} \PY{o}{+} \PY{n}{x13} \PY{o}{+} \PY{n}{x14} \PY{o}{==} \PY{l+m+mi}{120}
\PY{n}{prob} \PY{o}{+}\PY{o}{=} \PY{n}{x21} \PY{o}{+} \PY{n}{x22} \PY{o}{+} \PY{n}{x23} \PY{o}{+} \PY{n}{x24} \PY{o}{==} \PY{l+m+mi}{150}
\PY{n}{prob} \PY{o}{+}\PY{o}{=} \PY{n}{x31} \PY{o}{+} \PY{n}{x32} \PY{o}{+} \PY{n}{x33} \PY{o}{+} \PY{n}{x34} \PY{o}{==} \PY{l+m+mi}{200}

\PY{n}{prob} \PY{o}{+}\PY{o}{=} \PY{n}{x11} \PY{o}{+} \PY{n}{x21} \PY{o}{+} \PY{n}{x31} \PY{o}{==} \PY{l+m+mi}{100}
\PY{n}{prob} \PY{o}{+}\PY{o}{=} \PY{n}{x12} \PY{o}{+} \PY{n}{x22} \PY{o}{+} \PY{n}{x32} \PY{o}{==} \PY{l+m+mi}{60}
\PY{n}{prob} \PY{o}{+}\PY{o}{=} \PY{n}{x13} \PY{o}{+} \PY{n}{x23} \PY{o}{+} \PY{n}{x33} \PY{o}{==} \PY{l+m+mi}{130}
\PY{n}{prob} \PY{o}{+}\PY{o}{=} \PY{n}{x14} \PY{o}{+} \PY{n}{x24} \PY{o}{+} \PY{n}{x34} \PY{o}{==} \PY{l+m+mi}{180}

\PY{c+c1}{\PYZsh{} Solving problem}
\PY{n}{prob}\PY{o}{.}\PY{n}{solve}\PY{p}{(}\PY{p}{)}\PY{p}{;}
\end{Verbatim}
\end{tcolorbox}

    \begin{tcolorbox}[breakable, size=fbox, boxrule=1pt, pad at break*=1mm,colback=cellbackground, colframe=cellborder]
\prompt{In}{incolor}{2}{\boxspacing}
\begin{Verbatim}[commandchars=\\\{\}]
\PY{n+nb}{print}\PY{p}{(}\PY{l+s+s2}{\PYZdq{}}\PY{l+s+s2}{Status:}\PY{l+s+s2}{\PYZdq{}}\PY{p}{,} \PY{n}{LpStatus}\PY{p}{[}\PY{n}{prob}\PY{o}{.}\PY{n}{status}\PY{p}{]}\PY{p}{)}
\PY{n+nb}{print}\PY{p}{(}\PY{l+s+s2}{\PYZdq{}}\PY{l+s+s2}{Objective value: }\PY{l+s+s2}{\PYZdq{}}\PY{p}{,} \PY{n}{prob}\PY{o}{.}\PY{n}{objective}\PY{o}{.}\PY{n}{value}\PY{p}{(}\PY{p}{)}\PY{p}{)}

\PY{k}{for} \PY{n}{v} \PY{o+ow}{in} \PY{n}{prob}\PY{o}{.}\PY{n}{variables}\PY{p}{(}\PY{p}{)}\PY{p}{:}
    \PY{n+nb}{print}\PY{p}{(}\PY{n}{v}\PY{o}{.}\PY{n}{name}\PY{p}{,}\PY{l+s+s1}{\PYZsq{}}\PY{l+s+s1}{: }\PY{l+s+s1}{\PYZsq{}}\PY{p}{,} \PY{n}{v}\PY{o}{.}\PY{n}{value}\PY{p}{(}\PY{p}{)}\PY{p}{)}
\end{Verbatim}
\end{tcolorbox}

    \begin{Verbatim}[commandchars=\\\{\}]
Status: Optimal
Objective value:  2130.0
X11 :  60.0
X12 :  60.0
X13 :  0.0
X14 :  0.0
X21 :  0.0
X22 :  0.0
X23 :  0.0
X24 :  150.0
X31 :  40.0
X32 :  0.0
X33 :  130.0
X34 :  30.0
    \end{Verbatim}

    \hypertarget{optimization-with-pulp-round-2}{%
\subsubsection{Optimization with PuLP: Round
2!}\label{optimization-with-pulp-round-2}}

We now use set notation for this implementation

    \begin{tcolorbox}[breakable, size=fbox, boxrule=1pt, pad at break*=1mm,colback=cellbackground, colframe=cellborder]
\prompt{In}{incolor}{3}{\boxspacing}
\begin{Verbatim}[commandchars=\\\{\}]
\PY{k+kn}{from} \PY{n+nn}{pulp} \PY{k+kn}{import} \PY{o}{*}

\PY{n}{prob} \PY{o}{=} \PY{n}{LpProblem}\PY{p}{(}\PY{l+s+s1}{\PYZsq{}}\PY{l+s+s1}{Transportation\PYZus{}Problem}\PY{l+s+s1}{\PYZsq{}}\PY{p}{,} \PY{n}{LpMinimize}\PY{p}{)}


\PY{c+c1}{\PYZsh{} Sets}
\PY{n}{n\PYZus{}suppliers} \PY{o}{=} \PY{l+m+mi}{3}
\PY{n}{n\PYZus{}buyers} \PY{o}{=} \PY{l+m+mi}{4}

\PY{n}{I} \PY{o}{=} \PY{n+nb}{range}\PY{p}{(}\PY{n}{n\PYZus{}suppliers}\PY{p}{)}
\PY{n}{J} \PY{o}{=} \PY{n+nb}{range}\PY{p}{(}\PY{n}{n\PYZus{}buyers}\PY{p}{)}

\PY{n}{routes} \PY{o}{=} \PY{p}{[}\PY{p}{(}\PY{n}{i}\PY{p}{,} \PY{n}{j}\PY{p}{)} \PY{k}{for} \PY{n}{i} \PY{o+ow}{in} \PY{n}{I} \PY{k}{for} \PY{n}{j} \PY{o+ow}{in} \PY{n}{J}\PY{p}{]}


\PY{c+c1}{\PYZsh{} Parameters}
\PY{n}{costs} \PY{o}{=} \PY{p}{[}
    \PY{p}{[}\PY{l+m+mi}{4}\PY{p}{,} \PY{l+m+mi}{5}\PY{p}{,} \PY{l+m+mi}{6}\PY{p}{,} \PY{l+m+mi}{8}\PY{p}{]}\PY{p}{,}
    \PY{p}{[}\PY{l+m+mi}{4}\PY{p}{,} \PY{l+m+mi}{7}\PY{p}{,} \PY{l+m+mi}{9}\PY{p}{,} \PY{l+m+mi}{2}\PY{p}{]}\PY{p}{,} 
    \PY{p}{[}\PY{l+m+mi}{5}\PY{p}{,} \PY{l+m+mi}{8}\PY{p}{,} \PY{l+m+mi}{7}\PY{p}{,} \PY{l+m+mi}{6}\PY{p}{]}
\PY{p}{]}

\PY{n}{supply} \PY{o}{=} \PY{p}{[}\PY{l+m+mi}{120}\PY{p}{,} \PY{l+m+mi}{150}\PY{p}{,} \PY{l+m+mi}{200}\PY{p}{]}
\PY{n}{demand} \PY{o}{=} \PY{p}{[}\PY{l+m+mi}{100}\PY{p}{,} \PY{l+m+mi}{60}\PY{p}{,} \PY{l+m+mi}{130}\PY{p}{,} \PY{l+m+mi}{180}\PY{p}{]}



\PY{c+c1}{\PYZsh{} Variables}
\PY{n}{x} \PY{o}{=} \PY{n}{LpVariable}\PY{o}{.}\PY{n}{dicts}\PY{p}{(}\PY{l+s+s1}{\PYZsq{}}\PY{l+s+s1}{X}\PY{l+s+s1}{\PYZsq{}}\PY{p}{,} \PY{n}{routes}\PY{p}{,} \PY{n}{lowBound}\PY{o}{=}\PY{l+m+mi}{0}\PY{p}{)}

\PY{c+c1}{\PYZsh{} Objective}
\PY{n}{prob} \PY{o}{+}\PY{o}{=} \PY{n}{lpSum}\PY{p}{(}\PY{p}{[}\PY{n}{x}\PY{p}{[}\PY{n}{i}\PY{p}{,} \PY{n}{j}\PY{p}{]} \PY{o}{*} \PY{n}{costs}\PY{p}{[}\PY{n}{i}\PY{p}{]}\PY{p}{[}\PY{n}{j}\PY{p}{]} \PY{k}{for} \PY{n}{i} \PY{o+ow}{in} \PY{n}{I} \PY{k}{for} \PY{n}{j} \PY{o+ow}{in} \PY{n}{J}\PY{p}{]}\PY{p}{)}


\PY{c+c1}{\PYZsh{} Constraints}

\PY{c+c1}{\PYZsh{}\PYZsh{} Supply Constraints}
\PY{k}{for} \PY{n}{i} \PY{o+ow}{in} \PY{n+nb}{range}\PY{p}{(}\PY{n}{n\PYZus{}suppliers}\PY{p}{)}\PY{p}{:}
    \PY{n}{prob} \PY{o}{+}\PY{o}{=} \PY{n}{lpSum}\PY{p}{(}\PY{p}{[}\PY{n}{x}\PY{p}{[}\PY{n}{i}\PY{p}{,} \PY{n}{j}\PY{p}{]} \PY{k}{for} \PY{n}{j} \PY{o+ow}{in} \PY{n}{J}\PY{p}{]}\PY{p}{)} \PY{o}{==} \PY{n}{supply}\PY{p}{[}\PY{n}{i}\PY{p}{]}\PY{p}{,} \PY{l+s+sa}{f}\PY{l+s+s2}{\PYZdq{}}\PY{l+s+s2}{Supply}\PY{l+s+si}{\PYZob{}}\PY{n}{i}\PY{l+s+si}{\PYZcb{}}\PY{l+s+s2}{\PYZdq{}}
    
\PY{c+c1}{\PYZsh{}\PYZsh{} Demand Constraints}
\PY{k}{for} \PY{n}{j} \PY{o+ow}{in} \PY{n+nb}{range}\PY{p}{(}\PY{n}{n\PYZus{}buyers}\PY{p}{)}\PY{p}{:}
    \PY{n}{prob} \PY{o}{+}\PY{o}{=} \PY{n}{lpSum}\PY{p}{(}\PY{p}{[}\PY{n}{x}\PY{p}{[}\PY{n}{i}\PY{p}{,} \PY{n}{j}\PY{p}{]} \PY{k}{for} \PY{n}{i} \PY{o+ow}{in} \PY{n}{I}\PY{p}{]}\PY{p}{)} \PY{o}{==} \PY{n}{demand}\PY{p}{[}\PY{n}{j}\PY{p}{]}\PY{p}{,} \PY{l+s+sa}{f}\PY{l+s+s2}{\PYZdq{}}\PY{l+s+s2}{Demand}\PY{l+s+si}{\PYZob{}}\PY{n}{j}\PY{l+s+si}{\PYZcb{}}\PY{l+s+s2}{\PYZdq{}}
    
\PY{c+c1}{\PYZsh{} Solving problem}
\PY{n}{prob}\PY{o}{.}\PY{n}{solve}\PY{p}{(}\PY{p}{)}\PY{p}{;}
\end{Verbatim}
\end{tcolorbox}

    \begin{tcolorbox}[breakable, size=fbox, boxrule=1pt, pad at break*=1mm,colback=cellbackground, colframe=cellborder]
\prompt{In}{incolor}{4}{\boxspacing}
\begin{Verbatim}[commandchars=\\\{\}]
\PY{n+nb}{print}\PY{p}{(}\PY{l+s+s2}{\PYZdq{}}\PY{l+s+s2}{Status:}\PY{l+s+s2}{\PYZdq{}}\PY{p}{,} \PY{n}{LpStatus}\PY{p}{[}\PY{n}{prob}\PY{o}{.}\PY{n}{status}\PY{p}{]}\PY{p}{)}
\PY{n+nb}{print}\PY{p}{(}\PY{l+s+s2}{\PYZdq{}}\PY{l+s+s2}{Objective value: }\PY{l+s+s2}{\PYZdq{}}\PY{p}{,} \PY{n}{prob}\PY{o}{.}\PY{n}{objective}\PY{o}{.}\PY{n}{value}\PY{p}{(}\PY{p}{)}\PY{p}{)}

\PY{k}{for} \PY{n}{v} \PY{o+ow}{in} \PY{n}{prob}\PY{o}{.}\PY{n}{variables}\PY{p}{(}\PY{p}{)}\PY{p}{:}
    \PY{n+nb}{print}\PY{p}{(}\PY{n}{v}\PY{o}{.}\PY{n}{name}\PY{p}{,}\PY{l+s+s1}{\PYZsq{}}\PY{l+s+s1}{: }\PY{l+s+s1}{\PYZsq{}}\PY{p}{,} \PY{n}{v}\PY{o}{.}\PY{n}{value}\PY{p}{(}\PY{p}{)}\PY{p}{)}
\end{Verbatim}
\end{tcolorbox}

    \begin{Verbatim}[commandchars=\\\{\}]
Status: Optimal
Objective value:  2130.0
X\_(0,\_0) :  60.0
X\_(0,\_1) :  60.0
X\_(0,\_2) :  0.0
X\_(0,\_3) :  0.0
X\_(1,\_0) :  0.0
X\_(1,\_1) :  0.0
X\_(1,\_2) :  0.0
X\_(1,\_3) :  150.0
X\_(2,\_0) :  40.0
X\_(2,\_1) :  0.0
X\_(2,\_2) :  130.0
X\_(2,\_3) :  30.0
    \end{Verbatim}

    \hypertarget{changing-details-of-the-problem}{%
\subsection{Changing details of the
problem}\label{changing-details-of-the-problem}}

    \begin{tcolorbox}[breakable, size=fbox, boxrule=1pt, pad at break*=1mm,colback=cellbackground, colframe=cellborder]
\prompt{In}{incolor}{5}{\boxspacing}
\begin{Verbatim}[commandchars=\\\{\}]
\PY{n}{original\PYZus{}obj} \PY{o}{=} \PY{n}{prob}\PY{o}{.}\PY{n}{objective}
\PY{n}{val} \PY{o}{=} \PY{n}{prob}\PY{o}{.}\PY{n}{objective}\PY{o}{.}\PY{n}{value}\PY{p}{(}\PY{p}{)}
\PY{n}{r} \PY{o}{=} \PY{l+m+mf}{1.2}
\end{Verbatim}
\end{tcolorbox}

    \begin{tcolorbox}[breakable, size=fbox, boxrule=1pt, pad at break*=1mm,colback=cellbackground, colframe=cellborder]
\prompt{In}{incolor}{6}{\boxspacing}
\begin{Verbatim}[commandchars=\\\{\}]
\PY{n}{prob} \PY{o}{+}\PY{o}{=} \PY{n}{original\PYZus{}obj} \PY{o}{\PYZlt{}}\PY{o}{=} \PY{n}{r}\PY{o}{*}\PY{n}{val}\PY{p}{,} \PY{l+s+s2}{\PYZdq{}}\PY{l+s+s2}{Objective bound}\PY{l+s+s2}{\PYZdq{}}
\end{Verbatim}
\end{tcolorbox}

    \begin{tcolorbox}[breakable, size=fbox, boxrule=1pt, pad at break*=1mm,colback=cellbackground, colframe=cellborder]
\prompt{In}{incolor}{7}{\boxspacing}
\begin{Verbatim}[commandchars=\\\{\}]
\PY{n}{prob}
\end{Verbatim}
\end{tcolorbox}

            \begin{tcolorbox}[breakable, size=fbox, boxrule=.5pt, pad at break*=1mm, opacityfill=0]
\prompt{Out}{outcolor}{7}{\boxspacing}
\begin{Verbatim}[commandchars=\\\{\}]
Transportation\_Problem:
MINIMIZE
4*X\_(0,\_0) + 5*X\_(0,\_1) + 6*X\_(0,\_2) + 8*X\_(0,\_3) + 4*X\_(1,\_0) + 7*X\_(1,\_1) +
9*X\_(1,\_2) + 2*X\_(1,\_3) + 5*X\_(2,\_0) + 8*X\_(2,\_1) + 7*X\_(2,\_2) + 6*X\_(2,\_3) + 0
SUBJECT TO
Supply0: X\_(0,\_0) + X\_(0,\_1) + X\_(0,\_2) + X\_(0,\_3) = 120

Supply1: X\_(1,\_0) + X\_(1,\_1) + X\_(1,\_2) + X\_(1,\_3) = 150

Supply2: X\_(2,\_0) + X\_(2,\_1) + X\_(2,\_2) + X\_(2,\_3) = 200

Demand0: X\_(0,\_0) + X\_(1,\_0) + X\_(2,\_0) = 100

Demand1: X\_(0,\_1) + X\_(1,\_1) + X\_(2,\_1) = 60

Demand2: X\_(0,\_2) + X\_(1,\_2) + X\_(2,\_2) = 130

Demand3: X\_(0,\_3) + X\_(1,\_3) + X\_(2,\_3) = 180

Objective\_bound: 4 X\_(0,\_0) + 5 X\_(0,\_1) + 6 X\_(0,\_2) + 8 X\_(0,\_3)
 + 4 X\_(1,\_0) + 7 X\_(1,\_1) + 9 X\_(1,\_2) + 2 X\_(1,\_3) + 5 X\_(2,\_0) + 8 X\_(2,\_1)
 + 7 X\_(2,\_2) + 6 X\_(2,\_3) <= 2556

VARIABLES
X\_(0,\_0) Continuous
X\_(0,\_1) Continuous
X\_(0,\_2) Continuous
X\_(0,\_3) Continuous
X\_(1,\_0) Continuous
X\_(1,\_1) Continuous
X\_(1,\_2) Continuous
X\_(1,\_3) Continuous
X\_(2,\_0) Continuous
X\_(2,\_1) Continuous
X\_(2,\_2) Continuous
X\_(2,\_3) Continuous
\end{Verbatim}
\end{tcolorbox}
        
    \begin{tcolorbox}[breakable, size=fbox, boxrule=1pt, pad at break*=1mm,colback=cellbackground, colframe=cellborder]
\prompt{In}{incolor}{8}{\boxspacing}
\begin{Verbatim}[commandchars=\\\{\}]
\PY{c+c1}{\PYZsh{} Change the objective}
\PY{n}{prob} \PY{o}{+}\PY{o}{=} \PY{n}{x}\PY{p}{[}\PY{l+m+mi}{0}\PY{p}{,}\PY{l+m+mi}{0}\PY{p}{]}   \PY{c+c1}{\PYZsh{} minimize x[0,0]}
\end{Verbatim}
\end{tcolorbox}

    \begin{Verbatim}[commandchars=\\\{\}]
/opt/anaconda3/envs/python377/lib/python3.7/site-packages/pulp/pulp.py:1544:
UserWarning: Overwriting previously set objective.
  warnings.warn("Overwriting previously set objective.")
    \end{Verbatim}

    \begin{tcolorbox}[breakable, size=fbox, boxrule=1pt, pad at break*=1mm,colback=cellbackground, colframe=cellborder]
\prompt{In}{incolor}{9}{\boxspacing}
\begin{Verbatim}[commandchars=\\\{\}]
\PY{n}{prob}\PY{o}{.}\PY{n}{solve}\PY{p}{(}\PY{p}{)}
\end{Verbatim}
\end{tcolorbox}

            \begin{tcolorbox}[breakable, size=fbox, boxrule=.5pt, pad at break*=1mm, opacityfill=0]
\prompt{Out}{outcolor}{9}{\boxspacing}
\begin{Verbatim}[commandchars=\\\{\}]
1
\end{Verbatim}
\end{tcolorbox}
        
    \begin{tcolorbox}[breakable, size=fbox, boxrule=1pt, pad at break*=1mm,colback=cellbackground, colframe=cellborder]
\prompt{In}{incolor}{10}{\boxspacing}
\begin{Verbatim}[commandchars=\\\{\}]
\PY{n}{LpStatus}\PY{p}{[}\PY{n}{prob}\PY{o}{.}\PY{n}{status}\PY{p}{]}
\end{Verbatim}
\end{tcolorbox}

            \begin{tcolorbox}[breakable, size=fbox, boxrule=.5pt, pad at break*=1mm, opacityfill=0]
\prompt{Out}{outcolor}{10}{\boxspacing}
\begin{Verbatim}[commandchars=\\\{\}]
'Optimal'
\end{Verbatim}
\end{tcolorbox}
        
    \begin{tcolorbox}[breakable, size=fbox, boxrule=1pt, pad at break*=1mm,colback=cellbackground, colframe=cellborder]
\prompt{In}{incolor}{11}{\boxspacing}
\begin{Verbatim}[commandchars=\\\{\}]
\PY{n+nb}{print}\PY{p}{(}\PY{l+s+s2}{\PYZdq{}}\PY{l+s+s2}{Status:}\PY{l+s+s2}{\PYZdq{}}\PY{p}{,} \PY{n}{LpStatus}\PY{p}{[}\PY{n}{prob}\PY{o}{.}\PY{n}{status}\PY{p}{]}\PY{p}{)}
\PY{n+nb}{print}\PY{p}{(}\PY{l+s+s2}{\PYZdq{}}\PY{l+s+s2}{Objective value: }\PY{l+s+s2}{\PYZdq{}}\PY{p}{,} \PY{n}{prob}\PY{o}{.}\PY{n}{objective}\PY{o}{.}\PY{n}{value}\PY{p}{(}\PY{p}{)}\PY{p}{)}

\PY{k}{for} \PY{n}{v} \PY{o+ow}{in} \PY{n}{prob}\PY{o}{.}\PY{n}{variables}\PY{p}{(}\PY{p}{)}\PY{p}{:}
    \PY{n+nb}{print}\PY{p}{(}\PY{n}{v}\PY{o}{.}\PY{n}{name}\PY{p}{,}\PY{l+s+s1}{\PYZsq{}}\PY{l+s+s1}{: }\PY{l+s+s1}{\PYZsq{}}\PY{p}{,} \PY{n}{v}\PY{o}{.}\PY{n}{value}\PY{p}{(}\PY{p}{)}\PY{p}{)}
\end{Verbatim}
\end{tcolorbox}

    \begin{Verbatim}[commandchars=\\\{\}]
Status: Optimal
Objective value:  0.0
X\_(0,\_0) :  0.0
X\_(0,\_1) :  60.0
X\_(0,\_2) :  60.0
X\_(0,\_3) :  0.0
X\_(1,\_0) :  100.0
X\_(1,\_1) :  0.0
X\_(1,\_2) :  0.0
X\_(1,\_3) :  50.0
X\_(2,\_0) :  0.0
X\_(2,\_1) :  0.0
X\_(2,\_2) :  70.0
X\_(2,\_3) :  130.0
    \end{Verbatim}

    \begin{tcolorbox}[breakable, size=fbox, boxrule=1pt, pad at break*=1mm,colback=cellbackground, colframe=cellborder]
\prompt{In}{incolor}{12}{\boxspacing}
\begin{Verbatim}[commandchars=\\\{\}]
\PY{n}{original\PYZus{}obj}
\end{Verbatim}
\end{tcolorbox}

            \begin{tcolorbox}[breakable, size=fbox, boxrule=.5pt, pad at break*=1mm, opacityfill=0]
\prompt{Out}{outcolor}{12}{\boxspacing}
\begin{Verbatim}[commandchars=\\\{\}]
4*X\_(0,\_0) + 5*X\_(0,\_1) + 6*X\_(0,\_2) + 8*X\_(0,\_3) + 4*X\_(1,\_0) + 7*X\_(1,\_1) +
9*X\_(1,\_2) + 2*X\_(1,\_3) + 5*X\_(2,\_0) + 8*X\_(2,\_1) + 7*X\_(2,\_2) + 6*X\_(2,\_3) + 0
\end{Verbatim}
\end{tcolorbox}
        
    \begin{tcolorbox}[breakable, size=fbox, boxrule=1pt, pad at break*=1mm,colback=cellbackground, colframe=cellborder]
\prompt{In}{incolor}{13}{\boxspacing}
\begin{Verbatim}[commandchars=\\\{\}]
\PY{n}{original\PYZus{}obj}\PY{o}{.}\PY{n}{value}\PY{p}{(}\PY{p}{)}
\end{Verbatim}
\end{tcolorbox}

            \begin{tcolorbox}[breakable, size=fbox, boxrule=.5pt, pad at break*=1mm, opacityfill=0]
\prompt{Out}{outcolor}{13}{\boxspacing}
\begin{Verbatim}[commandchars=\\\{\}]
2430.0
\end{Verbatim}
\end{tcolorbox}
        
    \hypertarget{changing-constraint-coefficients}{%
\subsection{Changing Constraint
Coefficients}\label{changing-constraint-coefficients}}

    \begin{tcolorbox}[breakable, size=fbox, boxrule=1pt, pad at break*=1mm,colback=cellbackground, colframe=cellborder]
\prompt{In}{incolor}{14}{\boxspacing}
\begin{Verbatim}[commandchars=\\\{\}]
\PY{n}{a} \PY{o}{=} \PY{n}{prob}\PY{o}{.}\PY{n}{constraints}\PY{p}{[}\PY{l+s+s1}{\PYZsq{}}\PY{l+s+s1}{Supply0}\PY{l+s+s1}{\PYZsq{}}\PY{p}{]}
\end{Verbatim}
\end{tcolorbox}

    \begin{tcolorbox}[breakable, size=fbox, boxrule=1pt, pad at break*=1mm,colback=cellbackground, colframe=cellborder]
\prompt{In}{incolor}{15}{\boxspacing}
\begin{Verbatim}[commandchars=\\\{\}]
\PY{n}{a}\PY{o}{.}\PY{n}{changeRHS}\PY{p}{(}\PY{l+m+mi}{500}\PY{p}{)}
\end{Verbatim}
\end{tcolorbox}

    \begin{tcolorbox}[breakable, size=fbox, boxrule=1pt, pad at break*=1mm,colback=cellbackground, colframe=cellborder]
\prompt{In}{incolor}{16}{\boxspacing}
\begin{Verbatim}[commandchars=\\\{\}]
\PY{n}{a}
\end{Verbatim}
\end{tcolorbox}

            \begin{tcolorbox}[breakable, size=fbox, boxrule=.5pt, pad at break*=1mm, opacityfill=0]
\prompt{Out}{outcolor}{16}{\boxspacing}
\begin{Verbatim}[commandchars=\\\{\}]
1*X\_(0,\_0) + 1*X\_(0,\_1) + 1*X\_(0,\_2) + 1*X\_(0,\_3) + -500 = 0
\end{Verbatim}
\end{tcolorbox}
        
    \begin{tcolorbox}[breakable, size=fbox, boxrule=1pt, pad at break*=1mm,colback=cellbackground, colframe=cellborder]
\prompt{In}{incolor}{17}{\boxspacing}
\begin{Verbatim}[commandchars=\\\{\}]
\PY{n}{prob}\PY{o}{.}\PY{n}{constraints}\PY{p}{[}\PY{l+s+s1}{\PYZsq{}}\PY{l+s+s1}{Supply0}\PY{l+s+s1}{\PYZsq{}}\PY{p}{]}\PY{o}{.}\PY{n}{keys}\PY{p}{(}\PY{p}{)}
\end{Verbatim}
\end{tcolorbox}

            \begin{tcolorbox}[breakable, size=fbox, boxrule=.5pt, pad at break*=1mm, opacityfill=0]
\prompt{Out}{outcolor}{17}{\boxspacing}
\begin{Verbatim}[commandchars=\\\{\}]
odict\_keys([X\_(0,\_0), X\_(0,\_1), X\_(0,\_2), X\_(0,\_3)])
\end{Verbatim}
\end{tcolorbox}
       



 
    \hypertarget{multi-objective-optimization}{%
\section{Multi Objective
Optimization with PuLP}\label{multi-objective-optimization}}

We consider two objectives and compute the pareto efficient frontier. We
will implement the \(\epsilon\)-constraint method. That is, we will add
bounds based on an objective function and the optimize the alternate
objective function.

\hypertarget{transportation-problem}{%
\subsubsection{Transportation Problem}\label{transportation-problem}}

Sets: - \(J\) = set of demand nodes - \(I\) = set of supply nodes

Parameters: - \(D_j\): Demand at node \(j\) - \(S_i\): Supply from node
\(i\) - \(c_{ij}\): cost per unit to send supply \(i\) to demand \(j\)

Variables: - \(x_{ij}\): Transport volume from supply \(i\) to demand
\(j\) (units)

\begin{itemize}
\tightlist
\item
  Objective function:
  \[\min \left( obj1 = \sum_{i=1}^n\sum_{j=1}^mc_{ij}x_{ij}, \ \ \ \ \    obj2 =  x_{00} + x_{13} + x_{22} - x_{21} - x_{03}\right)\]
\item
  Constraints: \[\sum_{i=1}^nx_{ij}=S_i\] \[\sum_{i=1}^mx_{ij}=D_j\]
\item
  Decision variables: \[x_{ij} \geq 0 \ \ i \in I, j \in J\]
\end{itemize}

    \hypertarget{initial-optimization-with-pulp}{%
\subsubsection{Initial Optimization with
PuLP}\label{initial-optimization-with-pulp}}

    \begin{tcolorbox}[breakable, size=fbox, boxrule=1pt, pad at break*=1mm,colback=cellbackground, colframe=cellborder]
\prompt{In}{incolor}{1}{\boxspacing}
\begin{Verbatim}[commandchars=\\\{\}]
\PY{k+kn}{from} \PY{n+nn}{pulp} \PY{k+kn}{import} \PY{o}{*}

\PY{n}{prob} \PY{o}{=} \PY{n}{LpProblem}\PY{p}{(}\PY{l+s+s1}{\PYZsq{}}\PY{l+s+s1}{Transportation\PYZus{}Problem}\PY{l+s+s1}{\PYZsq{}}\PY{p}{,} \PY{n}{LpMinimize}\PY{p}{)}


\PY{c+c1}{\PYZsh{} Sets}
\PY{n}{n\PYZus{}suppliers} \PY{o}{=} \PY{l+m+mi}{3}
\PY{n}{n\PYZus{}buyers} \PY{o}{=} \PY{l+m+mi}{4}

\PY{n}{I} \PY{o}{=} \PY{n+nb}{range}\PY{p}{(}\PY{n}{n\PYZus{}suppliers}\PY{p}{)}
\PY{n}{J} \PY{o}{=} \PY{n+nb}{range}\PY{p}{(}\PY{n}{n\PYZus{}buyers}\PY{p}{)}

\PY{n}{routes} \PY{o}{=} \PY{p}{[}\PY{p}{(}\PY{n}{i}\PY{p}{,} \PY{n}{j}\PY{p}{)} \PY{k}{for} \PY{n}{i} \PY{o+ow}{in} \PY{n}{I} \PY{k}{for} \PY{n}{j} \PY{o+ow}{in} \PY{n}{J}\PY{p}{]}


\PY{c+c1}{\PYZsh{} Parameters}
\PY{n}{costs} \PY{o}{=} \PY{p}{[}
    \PY{p}{[}\PY{l+m+mi}{4}\PY{p}{,} \PY{l+m+mi}{5}\PY{p}{,} \PY{l+m+mi}{6}\PY{p}{,} \PY{l+m+mi}{8}\PY{p}{]}\PY{p}{,}
    \PY{p}{[}\PY{l+m+mi}{4}\PY{p}{,} \PY{l+m+mi}{7}\PY{p}{,} \PY{l+m+mi}{9}\PY{p}{,} \PY{l+m+mi}{2}\PY{p}{]}\PY{p}{,} 
    \PY{p}{[}\PY{l+m+mi}{5}\PY{p}{,} \PY{l+m+mi}{8}\PY{p}{,} \PY{l+m+mi}{7}\PY{p}{,} \PY{l+m+mi}{6}\PY{p}{]}
\PY{p}{]}

\PY{n}{supply} \PY{o}{=} \PY{p}{[}\PY{l+m+mi}{120}\PY{p}{,} \PY{l+m+mi}{150}\PY{p}{,} \PY{l+m+mi}{200}\PY{p}{]}
\PY{n}{demand} \PY{o}{=} \PY{p}{[}\PY{l+m+mi}{100}\PY{p}{,} \PY{l+m+mi}{60}\PY{p}{,} \PY{l+m+mi}{130}\PY{p}{,} \PY{l+m+mi}{180}\PY{p}{]}



\PY{c+c1}{\PYZsh{} Variables}
\PY{n}{x} \PY{o}{=} \PY{n}{LpVariable}\PY{o}{.}\PY{n}{dicts}\PY{p}{(}\PY{l+s+s1}{\PYZsq{}}\PY{l+s+s1}{X}\PY{l+s+s1}{\PYZsq{}}\PY{p}{,} \PY{n}{routes}\PY{p}{,} \PY{n}{lowBound}\PY{o}{=}\PY{l+m+mi}{0}\PY{p}{)}

\PY{c+c1}{\PYZsh{} Objectives}
\PY{n}{obj1} \PY{o}{=} \PY{n}{lpSum}\PY{p}{(}\PY{p}{[}\PY{n}{x}\PY{p}{[}\PY{n}{i}\PY{p}{,} \PY{n}{j}\PY{p}{]} \PY{o}{*} \PY{n}{costs}\PY{p}{[}\PY{n}{i}\PY{p}{]}\PY{p}{[}\PY{n}{j}\PY{p}{]} \PY{k}{for} \PY{n}{i} \PY{o+ow}{in} \PY{n}{I} \PY{k}{for} \PY{n}{j} \PY{o+ow}{in} \PY{n}{J}\PY{p}{]}\PY{p}{)}
\PY{n}{obj2} \PY{o}{=} \PY{n}{x}\PY{p}{[}\PY{l+m+mi}{0}\PY{p}{,}\PY{l+m+mi}{0}\PY{p}{]} \PY{o}{+} \PY{n}{x}\PY{p}{[}\PY{l+m+mi}{1}\PY{p}{,}\PY{l+m+mi}{3}\PY{p}{]} \PY{o}{+} \PY{n}{x}\PY{p}{[}\PY{l+m+mi}{2}\PY{p}{,}\PY{l+m+mi}{2}\PY{p}{]} \PY{o}{\PYZhy{}} \PY{n}{x}\PY{p}{[}\PY{l+m+mi}{2}\PY{p}{,}\PY{l+m+mi}{1}\PY{p}{]} \PY{o}{\PYZhy{}} \PY{n}{x}\PY{p}{[}\PY{l+m+mi}{0}\PY{p}{,}\PY{l+m+mi}{3}\PY{p}{]}

\PY{c+c1}{\PYZsh{}\PYZsh{} start with first objective}
\PY{n}{prob} \PY{o}{+}\PY{o}{=} \PY{n}{obj1}


\PY{c+c1}{\PYZsh{} Constraints}

\PY{c+c1}{\PYZsh{}\PYZsh{} Supply Constraints}
\PY{k}{for} \PY{n}{i} \PY{o+ow}{in} \PY{n+nb}{range}\PY{p}{(}\PY{n}{n\PYZus{}suppliers}\PY{p}{)}\PY{p}{:}
    \PY{n}{prob} \PY{o}{+}\PY{o}{=} \PY{n}{lpSum}\PY{p}{(}\PY{p}{[}\PY{n}{x}\PY{p}{[}\PY{n}{i}\PY{p}{,} \PY{n}{j}\PY{p}{]} \PY{k}{for} \PY{n}{j} \PY{o+ow}{in} \PY{n}{J}\PY{p}{]}\PY{p}{)} \PY{o}{==} \PY{n}{supply}\PY{p}{[}\PY{n}{i}\PY{p}{]}\PY{p}{,} \PY{l+s+sa}{f}\PY{l+s+s2}{\PYZdq{}}\PY{l+s+s2}{Supply}\PY{l+s+si}{\PYZob{}}\PY{n}{i}\PY{l+s+si}{\PYZcb{}}\PY{l+s+s2}{\PYZdq{}}
    
\PY{c+c1}{\PYZsh{}\PYZsh{} Demand Constraints}
\PY{k}{for} \PY{n}{j} \PY{o+ow}{in} \PY{n+nb}{range}\PY{p}{(}\PY{n}{n\PYZus{}buyers}\PY{p}{)}\PY{p}{:}
    \PY{n}{prob} \PY{o}{+}\PY{o}{=} \PY{n}{lpSum}\PY{p}{(}\PY{p}{[}\PY{n}{x}\PY{p}{[}\PY{n}{i}\PY{p}{,} \PY{n}{j}\PY{p}{]} \PY{k}{for} \PY{n}{i} \PY{o+ow}{in} \PY{n}{I}\PY{p}{]}\PY{p}{)} \PY{o}{==} \PY{n}{demand}\PY{p}{[}\PY{n}{j}\PY{p}{]}\PY{p}{,} \PY{l+s+sa}{f}\PY{l+s+s2}{\PYZdq{}}\PY{l+s+s2}{Demand}\PY{l+s+si}{\PYZob{}}\PY{n}{j}\PY{l+s+si}{\PYZcb{}}\PY{l+s+s2}{\PYZdq{}}
    
\PY{c+c1}{\PYZsh{} Solving problem}
\PY{n}{prob}\PY{o}{.}\PY{n}{solve}\PY{p}{(}\PY{p}{)}\PY{p}{;}
\end{Verbatim}
\end{tcolorbox}

    \begin{tcolorbox}[breakable, size=fbox, boxrule=1pt, pad at break*=1mm,colback=cellbackground, colframe=cellborder]
\prompt{In}{incolor}{2}{\boxspacing}
\begin{Verbatim}[commandchars=\\\{\}]
\PY{n+nb}{print}\PY{p}{(}\PY{l+s+s2}{\PYZdq{}}\PY{l+s+s2}{Status:}\PY{l+s+s2}{\PYZdq{}}\PY{p}{,} \PY{n}{LpStatus}\PY{p}{[}\PY{n}{prob}\PY{o}{.}\PY{n}{status}\PY{p}{]}\PY{p}{)}
\PY{n+nb}{print}\PY{p}{(}\PY{l+s+s2}{\PYZdq{}}\PY{l+s+s2}{Objective value: }\PY{l+s+s2}{\PYZdq{}}\PY{p}{,} \PY{n}{prob}\PY{o}{.}\PY{n}{objective}\PY{o}{.}\PY{n}{value}\PY{p}{(}\PY{p}{)}\PY{p}{)}

\PY{k}{for} \PY{n}{v} \PY{o+ow}{in} \PY{n}{prob}\PY{o}{.}\PY{n}{variables}\PY{p}{(}\PY{p}{)}\PY{p}{:}
    \PY{n+nb}{print}\PY{p}{(}\PY{n}{v}\PY{o}{.}\PY{n}{name}\PY{p}{,}\PY{l+s+s1}{\PYZsq{}}\PY{l+s+s1}{: }\PY{l+s+s1}{\PYZsq{}}\PY{p}{,} \PY{n}{v}\PY{o}{.}\PY{n}{value}\PY{p}{(}\PY{p}{)}\PY{p}{)}
\end{Verbatim}
\end{tcolorbox}

    \begin{Verbatim}[commandchars=\\\{\}]
Status: Optimal
Objective value:  2130.0
X\_(0,\_0) :  60.0
X\_(0,\_1) :  60.0
X\_(0,\_2) :  0.0
X\_(0,\_3) :  0.0
X\_(1,\_0) :  0.0
X\_(1,\_1) :  0.0
X\_(1,\_2) :  0.0
X\_(1,\_3) :  150.0
X\_(2,\_0) :  40.0
X\_(2,\_1) :  0.0
X\_(2,\_2) :  130.0
X\_(2,\_3) :  30.0
    \end{Verbatim}

    \begin{tcolorbox}[breakable, size=fbox, boxrule=1pt, pad at break*=1mm,colback=cellbackground, colframe=cellborder]
\prompt{In}{incolor}{3}{\boxspacing}
\begin{Verbatim}[commandchars=\\\{\}]
\PY{c+c1}{\PYZsh{} Record objective value}
\PY{n}{obj1\PYZus{}opt} \PY{o}{=} \PY{n}{obj1}\PY{o}{.}\PY{n}{value}\PY{p}{(}\PY{p}{)}
\PY{n}{obj1\PYZus{}opt}
\end{Verbatim}
\end{tcolorbox}

            \begin{tcolorbox}[breakable, size=fbox, boxrule=.5pt, pad at break*=1mm, opacityfill=0]
\prompt{Out}{outcolor}{3}{\boxspacing}
\begin{Verbatim}[commandchars=\\\{\}]
2130.0
\end{Verbatim}
\end{tcolorbox}
        
    \begin{tcolorbox}[breakable, size=fbox, boxrule=1pt, pad at break*=1mm,colback=cellbackground, colframe=cellborder]
\prompt{In}{incolor}{4}{\boxspacing}
\begin{Verbatim}[commandchars=\\\{\}]
\PY{c+c1}{\PYZsh{} Add both objective values to a list and also the solution}
\PY{n}{obj1\PYZus{}vals} \PY{o}{=} \PY{p}{[}\PY{n}{obj1}\PY{o}{.}\PY{n}{value}\PY{p}{(}\PY{p}{)}\PY{p}{]}
\PY{n}{obj2\PYZus{}vals} \PY{o}{=} \PY{p}{[}\PY{n}{obj2}\PY{o}{.}\PY{n}{value}\PY{p}{(}\PY{p}{)}\PY{p}{]}
\PY{n}{feasible\PYZus{}points} \PY{o}{=} \PY{p}{[}\PY{n}{prob}\PY{o}{.}\PY{n}{variables}\PY{p}{(}\PY{p}{)}\PY{p}{]}
\end{Verbatim}
\end{tcolorbox}

    \begin{tcolorbox}[breakable, size=fbox, boxrule=1pt, pad at break*=1mm,colback=cellbackground, colframe=cellborder]
\prompt{In}{incolor}{ }{\boxspacing}
\begin{Verbatim}[commandchars=\\\{\}]

\end{Verbatim}
\end{tcolorbox}

    \begin{tcolorbox}[breakable, size=fbox, boxrule=1pt, pad at break*=1mm,colback=cellbackground, colframe=cellborder]
\prompt{In}{incolor}{5}{\boxspacing}
\begin{Verbatim}[commandchars=\\\{\}]
\PY{c+c1}{\PYZsh{} Change objective functions and compute optimal objective value for obj2}
\PY{n}{prob} \PY{o}{+}\PY{o}{=} \PY{n}{obj2}
\PY{n}{prob}\PY{o}{.}\PY{n}{solve}\PY{p}{(}\PY{p}{)}

\PY{n}{obj2\PYZus{}opt} \PY{o}{=} \PY{n}{obj2}\PY{o}{.}\PY{n}{value}\PY{p}{(}\PY{p}{)}
\PY{n}{obj2\PYZus{}opt}
\end{Verbatim}
\end{tcolorbox}

    \begin{Verbatim}[commandchars=\\\{\}]
/opt/anaconda3/envs/python377/lib/python3.7/site-packages/pulp/pulp.py:1537:
UserWarning: Overwriting previously set objective.
  warnings.warn("Overwriting previously set objective.")
    \end{Verbatim}

            \begin{tcolorbox}[breakable, size=fbox, boxrule=.5pt, pad at break*=1mm, opacityfill=0]
\prompt{Out}{outcolor}{5}{\boxspacing}
\begin{Verbatim}[commandchars=\\\{\}]
-180.0
\end{Verbatim}
\end{tcolorbox}
        
    \begin{tcolorbox}[breakable, size=fbox, boxrule=1pt, pad at break*=1mm,colback=cellbackground, colframe=cellborder]
\prompt{In}{incolor}{6}{\boxspacing}
\begin{Verbatim}[commandchars=\\\{\}]
\PY{c+c1}{\PYZsh{} Append these values to the lists}
\PY{n}{obj1\PYZus{}vals}\PY{o}{.}\PY{n}{append}\PY{p}{(}\PY{n}{obj1}\PY{o}{.}\PY{n}{value}\PY{p}{(}\PY{p}{)}\PY{p}{)}
\PY{n}{obj2\PYZus{}vals}\PY{o}{.}\PY{n}{append}\PY{p}{(}\PY{n}{obj2}\PY{o}{.}\PY{n}{value}\PY{p}{(}\PY{p}{)}\PY{p}{)}
\PY{n}{feasible\PYZus{}points}\PY{o}{.}\PY{n}{append}\PY{p}{(}\PY{n}{prob}\PY{o}{.}\PY{n}{variables}\PY{p}{(}\PY{p}{)}\PY{p}{)}
\end{Verbatim}
\end{tcolorbox}

    \hypertarget{creating-the-pareto-efficient-frontier}{%
\subsection{Creating the Pareto Efficient
Frontier}\label{creating-the-pareto-efficient-frontier}}

    \begin{tcolorbox}[breakable, size=fbox, boxrule=1pt, pad at break*=1mm,colback=cellbackground, colframe=cellborder]
\prompt{In}{incolor}{7}{\boxspacing}
\begin{Verbatim}[commandchars=\\\{\}]
\PY{k+kn}{import} \PY{n+nn}{numpy} \PY{k}{as} \PY{n+nn}{np}

\PY{c+c1}{\PYZsh{} Create an inequality for objective 1}
\PY{n}{prob} \PY{o}{+}\PY{o}{=} \PY{n}{obj1} \PY{o}{\PYZlt{}}\PY{o}{=} \PY{n}{obj1\PYZus{}opt}\PY{p}{,} \PY{l+s+s2}{\PYZdq{}}\PY{l+s+s2}{Objective\PYZus{}bound1}\PY{l+s+s2}{\PYZdq{}}
\PY{n}{obj\PYZus{}constraint} \PY{o}{=} \PY{n}{prob}\PY{o}{.}\PY{n}{constraints}\PY{p}{[}\PY{l+s+s2}{\PYZdq{}}\PY{l+s+s2}{Objective\PYZus{}bound1}\PY{l+s+s2}{\PYZdq{}}\PY{p}{]}
\end{Verbatim}
\end{tcolorbox}

    \begin{tcolorbox}[breakable, size=fbox, boxrule=1pt, pad at break*=1mm,colback=cellbackground, colframe=cellborder]
\prompt{In}{incolor}{8}{\boxspacing}
\begin{Verbatim}[commandchars=\\\{\}]
\PY{c+c1}{\PYZsh{} Set to optimize objective 2}
\PY{n}{prob} \PY{o}{+}\PY{o}{=} \PY{n}{obj2}
\end{Verbatim}
\end{tcolorbox}

    \begin{Verbatim}[commandchars=\\\{\}]
/opt/anaconda3/envs/python377/lib/python3.7/site-packages/pulp/pulp.py:1537:
UserWarning: Overwriting previously set objective.
  warnings.warn("Overwriting previously set objective.")
    \end{Verbatim}

    \begin{tcolorbox}[breakable, size=fbox, boxrule=1pt, pad at break*=1mm,colback=cellbackground, colframe=cellborder]
\prompt{In}{incolor}{9}{\boxspacing}
\begin{Verbatim}[commandchars=\\\{\}]
\PY{c+c1}{\PYZsh{} Adjusting objective bound of objective 1}

\PY{n}{r\PYZus{}values} \PY{o}{=} \PY{n}{np}\PY{o}{.}\PY{n}{arange}\PY{p}{(}\PY{l+m+mi}{1}\PY{p}{,}\PY{l+m+mi}{2000}\PY{p}{,}\PY{l+m+mi}{10}\PY{p}{)}
\PY{k}{for} \PY{n}{r} \PY{o+ow}{in} \PY{n}{r\PYZus{}values}\PY{p}{:}
    \PY{n}{obj\PYZus{}constraint}\PY{o}{.}\PY{n}{changeRHS}\PY{p}{(}\PY{n}{r} \PY{o}{+} \PY{n}{obj1\PYZus{}opt}\PY{p}{)}
    \PY{k}{if} \PY{l+m+mi}{1} \PY{o}{==} \PY{n}{prob}\PY{o}{.}\PY{n}{solve}\PY{p}{(}\PY{p}{)}\PY{p}{:}
        \PY{n}{obj1\PYZus{}vals}\PY{o}{.}\PY{n}{append}\PY{p}{(}\PY{n}{obj1}\PY{o}{.}\PY{n}{value}\PY{p}{(}\PY{p}{)}\PY{p}{)}
        \PY{n}{obj2\PYZus{}vals}\PY{o}{.}\PY{n}{append}\PY{p}{(}\PY{n}{obj2}\PY{o}{.}\PY{n}{value}\PY{p}{(}\PY{p}{)}\PY{p}{)}
        \PY{n}{feasible\PYZus{}points}\PY{o}{.}\PY{n}{append}\PY{p}{(}\PY{n}{prob}\PY{o}{.}\PY{n}{variables}\PY{p}{(}\PY{p}{)}\PY{p}{)}

\PY{c+c1}{\PYZsh{} Remove objective 1 constraint}
\PY{n}{obj\PYZus{}constraint}\PY{o}{.}\PY{n}{changeRHS}\PY{p}{(}\PY{l+m+mi}{0}\PY{p}{)}
\PY{n}{obj\PYZus{}constraint}\PY{o}{.}\PY{n}{clear}\PY{p}{(}\PY{p}{)}
\end{Verbatim}
\end{tcolorbox}

    \begin{tcolorbox}[breakable, size=fbox, boxrule=1pt, pad at break*=1mm,colback=cellbackground, colframe=cellborder]
\prompt{In}{incolor}{10}{\boxspacing}
\begin{Verbatim}[commandchars=\\\{\}]
\PY{c+c1}{\PYZsh{} Create constaint for objective 2}
\PY{n}{prob} \PY{o}{+}\PY{o}{=} \PY{n}{obj2} \PY{o}{\PYZlt{}}\PY{o}{=} \PY{n}{obj2\PYZus{}opt}\PY{p}{,} \PY{l+s+s2}{\PYZdq{}}\PY{l+s+s2}{Objective\PYZus{}bound2}\PY{l+s+s2}{\PYZdq{}}
\PY{n}{obj2\PYZus{}constraint} \PY{o}{=} \PY{n}{prob}\PY{o}{.}\PY{n}{constraints}\PY{p}{[}\PY{l+s+s2}{\PYZdq{}}\PY{l+s+s2}{Objective\PYZus{}bound2}\PY{l+s+s2}{\PYZdq{}}\PY{p}{]}

\PY{c+c1}{\PYZsh{} set objective to objective 1}
\PY{n}{prob} \PY{o}{+}\PY{o}{=} \PY{n}{obj1}
\end{Verbatim}
\end{tcolorbox}

    \begin{tcolorbox}[breakable, size=fbox, boxrule=1pt, pad at break*=1mm,colback=cellbackground, colframe=cellborder]
\prompt{In}{incolor}{11}{\boxspacing}
\begin{Verbatim}[commandchars=\\\{\}]
\PY{c+c1}{\PYZsh{} Adjusting objective bound of objective 2}

\PY{n}{r\PYZus{}values} \PY{o}{=} \PY{n}{np}\PY{o}{.}\PY{n}{arange}\PY{p}{(}\PY{l+m+mi}{1}\PY{p}{,}\PY{l+m+mi}{400}\PY{p}{,}\PY{l+m+mi}{5}\PY{p}{)} \PY{c+c1}{\PYZsh{} may need to adjust this}
\PY{k}{for} \PY{n}{r} \PY{o+ow}{in} \PY{n}{r\PYZus{}values}\PY{p}{:}
    \PY{n}{obj2\PYZus{}constraint}\PY{o}{.}\PY{n}{changeRHS}\PY{p}{(}\PY{n}{r}\PY{o}{*}\PY{n}{obj2\PYZus{}opt}\PY{p}{)}
    \PY{k}{if} \PY{l+m+mi}{1} \PY{o}{==} \PY{n}{prob}\PY{o}{.}\PY{n}{solve}\PY{p}{(}\PY{p}{)}\PY{p}{:}
        \PY{n}{obj1\PYZus{}vals}\PY{o}{.}\PY{n}{append}\PY{p}{(}\PY{n}{obj1}\PY{o}{.}\PY{n}{value}\PY{p}{(}\PY{p}{)}\PY{p}{)}
        \PY{n}{obj2\PYZus{}vals}\PY{o}{.}\PY{n}{append}\PY{p}{(}\PY{n}{obj2}\PY{o}{.}\PY{n}{value}\PY{p}{(}\PY{p}{)}\PY{p}{)}
        \PY{n}{feasible\PYZus{}points}\PY{o}{.}\PY{n}{append}\PY{p}{(}\PY{n}{prob}\PY{o}{.}\PY{n}{variables}\PY{p}{(}\PY{p}{)}\PY{p}{)}

\PY{c+c1}{\PYZsh{} Remove objective 2 constraint}
\PY{n}{obj2\PYZus{}constraint}\PY{o}{.}\PY{n}{changeRHS}\PY{p}{(}\PY{l+m+mi}{0}\PY{p}{)}
\PY{n}{obj2\PYZus{}constraint}\PY{o}{.}\PY{n}{clear}\PY{p}{(}\PY{p}{)}
\end{Verbatim}
\end{tcolorbox}

    \begin{tcolorbox}[breakable, size=fbox, boxrule=1pt, pad at break*=1mm,colback=cellbackground, colframe=cellborder]
\prompt{In}{incolor}{12}{\boxspacing}
\begin{Verbatim}[commandchars=\\\{\}]
\PY{k+kn}{import} \PY{n+nn}{matplotlib}\PY{n+nn}{.}\PY{n+nn}{pyplot} \PY{k}{as} \PY{n+nn}{plt}
\PY{n}{plt}\PY{o}{.}\PY{n}{scatter}\PY{p}{(}\PY{n}{obj1\PYZus{}vals}\PY{p}{,} \PY{n}{obj2\PYZus{}vals}\PY{p}{)}
\PY{n}{plt}\PY{o}{.}\PY{n}{axvline}\PY{p}{(}\PY{n}{x}\PY{o}{=}\PY{n}{obj1\PYZus{}opt}\PY{p}{,} \PY{n}{color} \PY{o}{=} \PY{l+s+s1}{\PYZsq{}}\PY{l+s+s1}{y}\PY{l+s+s1}{\PYZsq{}}\PY{p}{)}
\PY{n}{plt}\PY{o}{.}\PY{n}{axhline}\PY{p}{(}\PY{n}{y}\PY{o}{=}\PY{n}{obj2\PYZus{}opt}\PY{p}{,} \PY{n}{color} \PY{o}{=} \PY{l+s+s1}{\PYZsq{}}\PY{l+s+s1}{y}\PY{l+s+s1}{\PYZsq{}}\PY{p}{)}
\PY{n}{plt}\PY{o}{.}\PY{n}{title}\PY{p}{(}\PY{l+s+s2}{\PYZdq{}}\PY{l+s+s2}{Pareto Efficient Frontier}\PY{l+s+s2}{\PYZdq{}}\PY{p}{)}
\PY{n}{plt}\PY{o}{.}\PY{n}{xlabel}\PY{p}{(}\PY{l+s+s2}{\PYZdq{}}\PY{l+s+s2}{Objective 1}\PY{l+s+s2}{\PYZdq{}}\PY{p}{)}
\PY{n}{plt}\PY{o}{.}\PY{n}{ylabel}\PY{p}{(}\PY{l+s+s2}{\PYZdq{}}\PY{l+s+s2}{Objective 2}\PY{l+s+s2}{\PYZdq{}}\PY{p}{)}
\end{Verbatim}
\end{tcolorbox}

            \begin{tcolorbox}[breakable, size=fbox, boxrule=.5pt, pad at break*=1mm, opacityfill=0]
\prompt{Out}{outcolor}{12}{\boxspacing}
\begin{Verbatim}[commandchars=\\\{\}]
Text(0, 0.5, 'Objective 2')
\end{Verbatim}
\end{tcolorbox}
        
    \begin{center}
    \adjustimage{max size={0.9\linewidth}{0.9\paperheight}}{transportation-multi-objective/pareto_frontier.png}
    \end{center}
    { \hspace*{\fill} \\}
    
    \hypertarget{comments}{%
\section{Comments}\label{comments}}

This code is a bit inefficient. It probably computes more pareto points
than needed.




\section{Jupyter Notebooks}
\begin{resource}
\begin{itemize}
\item \url{https://github.com/mathinmse/mathinmse.github.io/blob/master/Lecture-00B-Notebook-Basics.ipynb}
\item \url{https://github.com/mathinmse/mathinmse.github.io/blob/master/Lecture-00C-Writing-In-Jupyter.ipynb}
\end{itemize}
\end{resource}

\section{Reading and Writing}
\url{https://github.com/mathinmse/mathinmse.github.io/blob/master/Lecture-10B-Reading-and-Writing-Data.ipynb}

\section{Python Crash Course}
\url{https://github.com/rpmuller/PythonCrashCourse}


\section{Gurobi}
You can find lots great tutorial vidoes through Gurobi's youtube chanel.  

\todoSection{Link to introductory Gurobi files and installation instructions.
Show how grblogtools can show progress of integer programs and compare models across a set of instances.}

\subsection{Introductory Gurobi Examples}
\begin{enumerate}
\item \href{https://github.com/open-optimization/open-optimization-or-examples/blob/master/gurobi_helper_files/Gurobi-first-model.ipynb}{Gurobi-first-model.ipynb}
\item \href{https://github.com/open-optimization/open-optimization-or-examples/blob/master/gurobi_helper_files/Gurobi-knapsack1.ipynb}{Gurobi-knapsack1.ipynb}
\item \href{https://github.com/open-optimization/open-optimization-or-examples/blob/master/gurobi_helper_files/Gurobi-knapsack2.ipynb}{Gurobi-knapsack2.ipynb}
\item \href{https://github.com/open-optimization/open-optimization-or-examples/blob/master/gurobi_helper_files/Gurobi-knapsack3-Copy1.ipynb}{Gurobi-knapsack3.ipynb}
\item \href{https://github.com/open-optimization/open-optimization-or-examples/blob/master/gurobi_helper_files/Gurobi-knapsack-SOS1.ipynb}{Gurobi-knapsack-SOS1.ipynb}
\item \href{https://github.com/open-optimization/open-optimization-or-examples/blob/master/gurobi_helper_files/Gurobi-SOS2-piecewise-linear.ipynb}{Gurobi-SOS2-piecewise-linear.ipynb}
\end{enumerate}

\href{https://github.com/Gurobi}{Gurobi on GitHub}


\href{https://github.com/Gurobi/modeling-examples}{Gurobi Modeling Examples}

\href{https://github.com/Gurobi/grblogtools}{Gurobi Log Tools}


\section{Plots, Pandas, and Geopandas}
\todoSection{}
\subsection{Matplotlib.pyplot}
This is perhaps the most commonly used package for plotting.   The functionality is a lot like MatLab's plotting functionality.

\todo[inline]{Add Matplotlib tutorial reference.}


\subsection{Pandas}
Pandas is the most used package for manupulating data in python.   You can load data from excel, csv, json, or other types of files, and manipulate data in a variety of ways.
\todo[inline]{Add Panadas tutorial.}

\subsection{Geopandas}
Geopandas is a fantastic python package that allows you to interact with geospatial data, make plots, and analyze data.   This typically involves loading shapefiles that describe a region (or many regions) into a geopandas dataframe.  These function much like a regular pandas dataframe, but has a extra column for the geometry of the data and has other functionality for these types of dataframes.  

Here are some great tutorials.
\begin{itemize}
\item \href{https://jcutrer.com/python/learn-geopandas-plotting-usmaps}{tutorial}
\item \href{https://github.com/joncutrer/geopandas-tutorial}{tutorial github}
\end{itemize}
\section{Google OR Tools}
\todoSection{Give introduction to Google OR Tools and what problems it can solve.}
Google has been building their own set of optimization tools and marketing them to solve operations research problems.   They have a variety of type of solvers and have a number of example problems where they use the appropriate solver to solve a problem.   This is a constantly evolving set of tool. See the link below examples on vehicle routing, scheduling,  and more.  


\begin{center}
\fbox{
\href{https://developers.google.com/optimization}{Google OR Tools}}
\end{center}

\href{https://www.youtube.com/watch?v=AJ6LeiMe_PQ&t=757s&ab_channel=MixedIntegerProgramming}{Youtube!}
\href{https://www.youtube.com/watch?v=iF2rHY318iU&ab_channel=MixedIntegerProgramming}{Video!}

