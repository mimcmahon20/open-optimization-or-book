\begin{examplewithallcode}{Camping Trip}
    {}
    {https://github.com/open-optimization/open-optimization-or-book/tree/master/Intro-Math-Programming/baseText/optimization/modeling-linear-programming/examples/knapsack/knapsack-gurobi.ipynb}
    {https://github.com/open-optimization/open-optimization-or-book/tree/master/Intro-Math-Programming/baseText/optimization/modeling-linear-programming/examples/knapsack/knapsack-pulp.ipynb}
    Imagine you are preparing for a week-long camping trip to the mountains. You have a backpack with a weight capacity of 20 kilograms. Your goal is to pack the most valuable items to ensure a comfortable and safe trip without exceeding the backpack's weight limit. Each item has a weight and a value associated with it, representing its importance and utility for the trip. The table below lists the potential items you can take, their weights in kilograms, and their values on a scale from 1 to 10 (with 10 being the most valuable).
    \newline
    Formulate an Integer Program to find the optimal items you should bring on your trip with you. 
    \end{examplewithallcode}
    
    \begin{table}[h!] \begin{center} \begin{tabular} {|l|l|l|l|} 
    \hline Item & Index & Weight (kg) & Value  \\ \hline
    \hline Tent & A & 4 & 10  \\
    \hline Stove & B & 2 & 9  \\
    \hline Sleeping Bag & C & 3 & 8  \\
    \hline First Aid Kit & D & 1 & 7  \\
    \hline Food Supplies & E & 5 & 9  \\
    \hline Water Filter & F & 1 & 6  \\
    \hline Clothes & G & 4 & 5  \\
    \hline Map and Compass & H & 0.5 & 7  \\
    \hline \end{tabular} \end{center} \end{table}
    
    \begin{solution}
    \underline{Decision variables:} \\
    $x_i$ : whether to include item $i$ in the backpack, where $i \in \{A, B, C, D, E, F, G, H\}$, $x_i = 1$ if item $i$ is included, and $x_i = 0$ otherwise.
    
    \smallskip \underline{Objective and Constraints:}
    \begin{align*}
    \mbox{Max~~ } & Z = 10x_A + 9x_B + 8x_C + 7x_D + 9x_E + 6x_F + 5x_G + 7x_H  \\
    \mbox{s.t.~~} & 4x_A + 2x_B + 3x_C + 1x_D + 5x_E + 1x_F + 4x_G + 0.5x_H \le 20 \\
    & x_A, x_B, x_C, x_D, x_E, x_F, x_G, x_H \in \{0,1\}.
    \end{align*}
    \end{solution}
    
    \todo{inline}{Add excercise for user: Suppose you were shipped the wrong bag, and your carrying limit is now only 15 kilograms. What changes would you make to your plan considering this change? What changes would you make to the model to reflect this change? What're the new optimal items to carry?}
    