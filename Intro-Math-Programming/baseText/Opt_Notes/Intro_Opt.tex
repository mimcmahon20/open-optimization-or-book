%By Douglas Bish
%The text is licensed under the
%\href{http://creativecommons.org/licenses/by-sa/4.0/}{Creative Commons
%Attribution-ShareAlike 4.0 International License}.
%
%This file has been modified by Robert Hildebrand 2020.  
%CC BY SA 4.0 licence still applies.


Optimization (i.e., Mathematical Programming) seeks to select, from a set of alternative solutions (decisions), a solution that is ``best" for a given performance criteria (i.e., maximize or minimizes the criteria). The following is a general optimization problem: 
$$ \text{max} \{f(\mathbf{x},\mathbf{y}): A(\mathbf{x)} + G(\mathbf{y}) \le \mathbf{b}_1, H(\mathbf{x)} + W(\mathbf{y}) =  \mathbf{b}_2, \mathbf{x} \in \mathcal{Z^+}, \mathbf{y} \in \mathcal{R^+}\},$$ 

where $\mathbf{x}$ and $\mathbf{y}$ are vectors of decision variables, $f(\mathbf{x},\mathbf{y})$ is the {\it objective function}, which defines the ``best'' solution (in this case the optimization problem seeks to maximize the objective function), and $A(\mathbf{x)} + G(\mathbf{y}) \le \mathbf{b}_1$, $H(\mathbf{x)} + W(\mathbf{y}) = \mathbf{b}_2$, $\mathbf{x} \in \mathcal{Z^+}$, and $\mathbf{y} \in \mathcal{R^+}$ are the {\it constraints} that define the set of possible solutions.


\bigskip Depending on the nature of the objective function, the constraints, and the input parameters, we can make some broad classifications of optimization problems, as follows:

\bigskip \underline{\bf Linear Optimization:} Linear optimization, i.e., a linear program (LP), has a linear objective function subject to a set of linear constraints and continuous decision variables.\\

 {\bf Definition:} A function $f(x_1,x_2,\cdots,x_n)$ is linear if, and only if, we have $f(x_1,x_2,\cdots,x_n) = c_1x_1 + c_2x_2 + \cdots + c_nx_n$, where the  $c_1,c_2,\cdots,c_n$ coefficients are constants.  \\

An LP has the following general form:
$$ \text{max} \{\mathbf{cx}: \mathbf{A}\mathbf{x} = \mathbf{b},  x \in \mathcal{R}\},$$ where $\mathbf{x}$ is a vector of decision variables, and the vectors $\mathbf{c}$ and $\mathbf{b}$, as well as the matrix $\mathbf{A}$, are constant problem parameters.

\bigskip \underline{\bf Nonlinear Optimization:} Nonlinear optimization, i.e., a nonlinear program, is similar to an LP, but objective function and/or the constraints are nonlinear. \\

\underline{\bf Integer Optimization:} Integer optimization, , i.e., an integer program (IP), is much like an LP, but some) variables restricted to take only integer values. \\
 
% Add others, ...
% Dynamic Program - Optimize a multi-stage problem, which is often recursive in nature.
% Stochastic Program - Optimize a problem where some of the input data is stochastic.
% Network Optimization - Optimize problems with special network structures.



\bigskip To use optimization, first you must formulate your model, based on the system of interest and any simplifications required (i.e., assumptions).  Formulating the model is not enough, we are also interested in solving the problem, and in a reasonable amount of time (however that is determined). To solve these problems, algorithms are developed.  An algorithms is a step-by-step process for finding a solution.  We can broadly define different types of algorithms as follows:
\begin{itemize}
\item Optimal algorithms - processes that solve the model to optimality (and proves optimality).
\item Near-optimal algorithms with bounds (heuristics), processes that do not guarantee optimality, but provides ``good" solutions with known bounds.
\item Other heuristic algorithms, processes that provide a ``good" solution, but bounds are not provided, or are not that useful.
\end{itemize}

Algorithms can also be categorized based on performance, for instance, usually a {\it polynomial time algorithm} is better than an {\it exponential time algorithm}.

\bigskip The class will almost exclusively focus on Linear Programs (LP) because: 1) LP are useful for many problems; 2) LPs are, relatively, easy to solve; and most importantly 3) LP are an important foundation for further courses in optimization.

\subsection{Notation}

\begin{itemize}
\item We use bold text to indicate a matrix or vector, e.g., the matrix ${\bf A}$ or the vector ${\bf x}$.
\end{itemize}