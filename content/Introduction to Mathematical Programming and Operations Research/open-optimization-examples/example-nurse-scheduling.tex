% Copywrite Robert Hildeband 2019


\begin{example}{Nurse Scheduling}{--- Adapted from "Applied Integer Programming". exercise 2.6 and 2.7}
\end{example}
Nurses in large hospitals usually work 3 days a week. Daily demand for nurses is summarized in Table 2.4. Determine the number of nurses required per schedule type so that the total wage cost is minimized.
\begin{enumerate}
\item  Use the numbers (not symbols) in the table to model this problem instance.
\item Solve this instance and submit the code.
\item Let $S$ denote the set of schedule types.  How many variables of each type (continuous, binary, integer) does your model have in terms of the number of schedules $|S|$?
\item If part-time nurses are hired at the rate of $\$175$/day, formulate the problem to minimize the total cost. 
\item If part-time nurses must be accompanied by at least three full-time nurses, how would you formulate this constraint?
\end{enumerate}
\begin{table}[h]
\begin{center}
\begin{tabular}{l|ccccc|c}
\hline
\hline
& & & Schedule Type & & \\
\hline
Day & 1 & 2 & 3 & 4 & 5 & Nurses Required\\
\hline
Monday & X & & X & & & 20\\
Tuesday & X & &  &  & X & 25\\
Wednesday & & & X & X & & 26\\
Thursday& & X & & & X& 26\\
Friday & & X & & X & X & 30\\
Saturday& & X& & X& & 30\\
Sunday& X& & X & && 35\\
\hline
Weekly wage& 525 & 470 & 550 & 500 & 425\\
\hline 
\hline
\end{tabular}
\end{center}
\end{table}

\begin{itemize}
		\item Let 
		$$
		x_i = \textrm{The number of nurses to assign to schedule } i\ \forall\ i \in \left\lbrace 1,2,...,5 \right\rbrace.
		$$
\end{itemize}	
		\vspace{.25in}
		The model is
		\begin{align*}
		\textrm{Minimize}	\hspace{.25in}	525x_1+470x_2+550x_3+500x_4+425x_5&&&&
		\end{align*}
		\vspace{-.43in}
		\begin{align*}
		\textrm{s.t.} 		\hspace{.6in}	x_1+x_3 &\geq 20							\\
											x_1+x_5 &\geq 25							\\
											x_3+x_4 &\geq 26							\\
											x_2+x_5 &\geq 26							\\
										x_2+x_4+x_5 &\geq 30							\\
											x_2+x_4 &\geq 30							\\
											x_1+x_3 &\geq 35							\\
												x_i &\in \mathbb{Z}_+ & &\forall\ i \in \left\lbrace 1,2,...,5 \right\rbrace
		\end{align*}
		
		The table tells us which schedule types contribute to each day's staff; i.e. Mondays are staffed entirely by nurses assigned to schedule types 1 and 3, Tuesdays by those assigned to type 1 and 5, and so on. At least 20 nurses are required to be on duty on Mondays; so the number of nurses assigned to schedules 1 and 3 should be at least 20. That is,
			$$
			x_1+x_3 \geq 20.
			$$
		Extending this logic to each day gives us all seven constraints. \\
		
		Alternately, by replacing the X's in the table with 1's we are left with a constraint matrix $S$. Let $\vec{x}^\top = [x_1,x_2,x_3,x_4,x_5]$ be the decision variables and $\vec{r}^\top=[20,25,26,26,30,30,35]$ describe the nurses required for each day, so that
			$$
			S\ \vec{x} \geq \vec{r}
			$$
		describes all seven constraints. 
		%This is the approach taken in the sample code below.
